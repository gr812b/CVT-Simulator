\documentclass[12pt, titlepage]{article}

\usepackage{booktabs}
\usepackage{tabularx}
\usepackage{hyperref}
\hypersetup{
    colorlinks,
    citecolor=black,
    filecolor=black,
    linkcolor=red,
    urlcolor=blue
}
\usepackage[round]{natbib}

\input{../Comments}
%% Common Parts

\newcommand{\progname}{Baja Dynamics} % PUT YOUR PROGRAM NAME HERE
\newcommand{\authname}{Team \#17, Team Name
\\ Grace McKenna
\\ Travis Wing
\\ Cameron Dunn
\\ Kai Arseneau} % AUTHOR NAMES                  

\usepackage{hyperref}
    \hypersetup{colorlinks=true, linkcolor=blue, citecolor=blue, filecolor=blue,
                urlcolor=blue, unicode=false}
    \urlstyle{same}
                                


\begin{document}

\title{Verification and Validation Report: \progname} 
\author{\authname}
\date{\today}
	
\maketitle

\pagenumbering{roman}

\section{Revision History}

\begin{tabularx}{\textwidth}{p{3cm}p{2cm}X}
\toprule {\bf Date} & {\bf Version} & {\bf Notes}\\
\midrule
Date 1 & 1.0 & Notes\\
Date 2 & 1.1 & Notes\\
\bottomrule
\end{tabularx}

~\newpage

\section{Symbols, Abbreviations and Acronyms}

\renewcommand{\arraystretch}{1.2}
\begin{tabular}{l l} 
  \toprule		
  \textbf{symbol} & \textbf{description}\\
  \midrule 
  T & Test\\
  \bottomrule
\end{tabular}\\

\wss{symbols, abbreviations or acronyms -- you can reference the SRS tables if needed}

\newpage

\tableofcontents

\listoftables %if appropriate

\listoffigures %if appropriate

\newpage

\pagenumbering{arabic}

This document ...

\section{Functional Requirements Evaluation}

\section{Nonfunctional Requirements Evaluation}

This section will cover the evaluation of the Nonfunctional Requirements. 

\subsection{Accuracy}
		
\subsection{Usability}
The Usability/Understandability survey remains in progress at this time and results will be discussed in the \href{file:../UsabilityReport/UsabilityReport.pdf}{Usability Report}. 
Therefore, Usability test-1 and Usability test-2 have not been fully completed yet, however as they are in progress these tests will be completed as future work. 

\begin{enumerate}
\item{\textbf{test-1}: Navigating Main Interface}\\
Survey question: On a scale of 1-5 with 1 being extremely difficult and 5 being extremely easy, how easy was it to navigate the main interface? 
\item{\textbf{test-2}: Use of Most Common Features}\\
Survey question: For the following main features: Inputting parameters, Adjusting parameters, Viewing data outputs, Saving and exporting data. 
Rate each feature on a scale of 1-5 with 1 being extremely difficult and 5 being extremely easy
\end{enumerate}

\subsection{Maintainability}

\subsection{Verifiability}

\subsection{Understandability}
The Usability/Understandability survey remains in progress at this time and results will be discussed in the \href{file:../UsabilityReport/UsabilityReport.pdf}{Usability Report}. 
Therefore, Understandability test-1 and Understandability test-2 have not been completed yet, however as they are in progress these tests will completed as future work. 

\begin{enumerate}
  \item{\textbf{test-1}: Function Purpose}\\
  Survey question: For the following main features: Inputting parameters, Adjusting parameters, Viewing data outputs, Saving and exporting data. 
  Was the purpose of each function clear, on a scale of 1-5 with 1 being very unclear and 5 being extremely clear. 
  \item{\textbf{test-2}: Understanding Simulation Outputs}\\
  Survey question: On a scale of 1-5, with 1 being very unclear and 5 being extremely clear, how well did you understand the simulation results and output?
\end{enumerate}

\subsection{Reusability}
	
\section{Comparison to Existing Implementation}	

This section will not be appropriate for every project.

\section{Unit Testing}

\section{Changes Due to Testing}

\wss{This section should highlight how feedback from the users and from 
the supervisor (when one exists) shaped the final product.  In particular 
the feedback from the Rev 0 demo to the supervisor (or to potential users) 
should be highlighted.}

\section{Automated Testing}
		
\section{Trace to Requirements}
		
\section{Trace to Modules}		

\section{Code Coverage Metrics}

\bibliographystyle{plainnat}
\bibliography{../../refs/References}

\newpage{}
\section*{Appendix --- Reflection}

The information in this section will be used to evaluate the team members on the
graduate attribute of Reflection.

\input{../Reflection.tex}

\begin{enumerate}
  \item What went well while writing this deliverable? 
  \item What pain points did you experience during this deliverable, and how
    did you resolve them?
  \item Which parts of this document stemmed from speaking to your client(s) or
  a proxy (e.g. your peers)? Which ones were not, and why?
  \item In what ways was the Verification and Validation (VnV) Plan different
  from the activities that were actually conducted for VnV?  If there were
  differences, what changes required the modification in the plan?  Why did
  these changes occur?  Would you be able to anticipate these changes in future
  projects?  If there weren't any differences, how was your team able to clearly
  predict a feasible amount of effort and the right tasks needed to build the
  evidence that demonstrates the required quality?  (It is expected that most
  teams will have had to deviate from their original VnV Plan.)
\end{enumerate}

\end{document}