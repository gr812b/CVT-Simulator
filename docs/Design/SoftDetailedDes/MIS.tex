\documentclass[12pt, titlepage]{article}

\usepackage{amsmath, mathtools}

\usepackage[round]{natbib}
\usepackage{amsfonts}
\usepackage{amssymb}
\usepackage{graphicx}
\usepackage{colortbl}
\usepackage{xr}
\usepackage{hyperref}
\usepackage{longtable}
\usepackage{xfrac}
\usepackage{tabularx}
\usepackage{float}
\usepackage{siunitx}
\usepackage{booktabs}
\usepackage{multirow}
\usepackage[section]{placeins}
\usepackage{caption}
\usepackage{fullpage}

\hypersetup{
bookmarks=true,     % show bookmarks bar?
colorlinks=true,       % false: boxed links; true: colored links
linkcolor=red,          % color of internal links (change box color with linkbordercolor)
citecolor=blue,      % color of links to bibliography
filecolor=magenta,  % color of file links
urlcolor=cyan          % color of external links
}

\usepackage{array}

\externaldocument{../../SRS/SRS}

\input{../../Comments}
%% Common Parts

\newcommand{\progname}{Baja Dynamics} % PUT YOUR PROGRAM NAME HERE
\newcommand{\authname}{Team \#17, Team Name
\\ Grace McKenna
\\ Travis Wing
\\ Cameron Dunn
\\ Kai Arseneau} % AUTHOR NAMES                  

\usepackage{hyperref}
    \hypersetup{colorlinks=true, linkcolor=blue, citecolor=blue, filecolor=blue,
                urlcolor=blue, unicode=false}
    \urlstyle{same}
                                


\begin{document}

\title{Module Interface Specification for \progname{}}

\author{\authname}

\date{\today}

\maketitle

\pagenumbering{roman}

\section{Revision History}

\begin{tabularx}{\textwidth}{p{3cm}p{2cm}X}
\toprule {\bf Date} & {\bf Version} & {\bf Notes}\\
\midrule
Date 1 & 1.0 & Notes\\
Date 2 & 1.1 & Notes\\
\bottomrule
\end{tabularx}

~\newpage

\section{Symbols, Abbreviations and Acronyms}

See SRS Documentation at \wss{give url}

\wss{Also add any additional symbols, abbreviations or acronyms}

\newpage

\tableofcontents

\newpage

\pagenumbering{arabic}

\section{Introduction}

The following document details the Module Interface Specifications for
the \progname program which is designed for optimizing McMaster Baja vehicles. 
This document specifies how each module interacts with one another throughout the program. 

Complementary documents include the System Requirement Specifications
and Module Guide.  The full documentation and implementation can be
found at \url{https://github.com/gr812b/CVT-Simulator}.

\section{Notation}

\wss{You should describe your notation.  You can use what is below as
  a starting point.}

The structure of the MIS for modules comes from \citet{HoffmanAndStrooper1995},
with the addition that template modules have been adapted from
\cite{GhezziEtAl2003}.  The mathematical notation comes from Chapter 3 of
\citet{HoffmanAndStrooper1995}.  For instance, the symbol := is used for a
multiple assignment statement and conditional rules follow the form $(c_1
\Rightarrow r_1 | c_2 \Rightarrow r_2 | ... | c_n \Rightarrow r_n )$.

The following table summarizes the primitive data types used by \progname. 

\begin{center}
\renewcommand{\arraystretch}{1.2}
\noindent 
\begin{tabular}{l l p{7.5cm}} 
\toprule 
\textbf{Data Type} & \textbf{Notation} & \textbf{Description}\\ 
\midrule
character & char & a single symbol or digit\\
integer & $\mathbb{Z}$ & a number without a fractional component in (-$\infty$, $\infty$) \\
natural number & $\mathbb{N}$ & a number without a fractional component in [1, $\infty$) \\
real & $\mathbb{R}$ & any number in (-$\infty$, $\infty$)\\
positive real & $\mathbf{R}_{+}$ & any real number ($\mathbf{R}$) in ($0$, $\infty$) \\
input & $\mathbb{I}$ & a set of values \{$\mathbf{R}_{+}$, $\mathbb{R} \rightarrow \mathbb{R}$, $\mathbf{R}_{+}$, $\mathbf{R}_{+}$, $\mathbb{R} \rightarrow \mathbb{R}$, $\mathbf{R}_{+}$,$\mathbf{R}_{+}$, $\mathbf{R}_{+}$, $\mathbf{R}_{+}$, $\mathbf{R}_{+}$, $\mathbf{R}_{+}$\} that represent the input of the program \\
state & $\mathbb{S}$ & a set of values \{\} representing the state of the simulation \\
\bottomrule
\end{tabular} 
\end{center}

\noindent
The specification of \progname \ uses some derived data types: sequences, strings, and
tuples. Sequences are lists filled with elements of the same data type. Strings
are sequences of characters. Tuples contain a list of values, potentially of
different types. In addition, \progname \ uses functions, which
are defined by the data types of their inputs and outputs. Local functions are
described by giving their type signature followed by their specification.

\section{Module Decomposition}

The following table is taken directly from the Module Guide document for this project.

\begin{table}[h!]
\centering
\begin{tabular}{p{0.3\textwidth} p{0.6\textwidth}}
\toprule
\textbf{Level 1} & \textbf{Level 2}\\
\midrule

{Hardware-Hiding Module} & ~ \\
\midrule

\multirow{7}{0.3\textwidth}{Behaviour-Hiding Module}
& Engine Simulator Module\\
& External Forces Module\\
& CVT Simulation Module\\
& Input Module\\
& ODE Solver Module\\
& Main Module\\ 
& Playback Module\\
& Visualizer Module\\
& Constants Module\\
& State Module\\
& Backend Controller Module\\
\midrule

\multirow{3}{0.3\textwidth}{Software Decision Module}
& GUI Module\\
& File Output Module\\
& Communication Module\\
\bottomrule

\end{tabular}
\caption{Module Hierarchy}
\label{TblMH}
\end{table}

\newpage
~\newpage

\section{Engine Simulator Module} \label{Module_engine_simulator} \wss{Use labels for
  cross-referencing}

\wss{You can reference SRS labels, such as R\ref{R_Inputs}.}

\wss{It is also possible to use \LaTeX for hypperlinks to external documents.}

\subsection{Module}

Engine Module

\subsection{Uses}

\begin{itemize}
  \item Constants Module (\ref{Module_constants})
\end{itemize}

\subsection{Syntax}

\subsubsection{Exported Constants}
None

\subsubsection{Exported Access Programs}

\begin{center}
\begin{tabular}{p{4cm} p{4cm} p{4cm} p{2cm}}
\hline
\textbf{Name} & \textbf{In} & \textbf{Out} & \textbf{Exceptions} \\
\hline
getTorque & angularVeloctiy ($\mathbb{R}$) & torque ($\mathbb{R}$) & - \\
calcuAngularAccel & angularVeloctiy ($\mathbb{R}$), loadTorque ($\mathbb{R}$)  & angularAcceleration ($\mathbb{R}$) & - \\
\hline
\end{tabular}
\end{center}

\subsection{Semantics}

\subsubsection{State Variables}

\begin{itemize}
  \item Torque curve $\mathbb{R}$ $\rightarrow$ $\mathbb{R}$
  \item Inertia $\mathbb{R}$
\end{itemize}

\subsubsection{Environment Variables}

None

\subsubsection{Assumptions}
\begin{itemize}
  \item Torque Curve is initialized from the constants module
  \item Inertia is positive
\end{itemize}

\subsubsection{Access Routine Semantics}

\noindent getTorque(angularVeloctiy):
\begin{itemize}
\item output: torque:= torqueCurve(angularVeloctiy) 
\end{itemize}

\noindent calcAngularAccel(angularVeloctiy, loadTorque):
\begin{itemize}
\item output: angularAcceleration:= (loadTorque - getTorque(angularVeloctiy))/inertia
\end{itemize}

\subsubsection{Local Functions}

None

\newpage

\section{External Forces Module} \label{Module_external_forces} 

\subsection{Module}

Load Simulator

\subsection{Uses}

\begin{itemize}
  \item Constants Module (\ref{Module_constants})
\end{itemize}

\subsection{Syntax}

\subsubsection{Exported Constants}
None

\subsubsection{Exported Access Programs}

\begin{center}
\begin{tabular}{p{4cm} p{4cm} p{4cm} p{2cm}}
\hline
\textbf{Name} & \textbf{In} & \textbf{Out} & \textbf{Exceptions} \\
\hline
calcInclineForce & - & inclineForce $\mathbb{F}$ & - \\
calcDragForce & velocity $\mathbb{F}$ & dragForce $\mathbb{F}$ & - \\
calcLoadTorque & velocity $\mathbb{F}$ & loadTorque $\mathbb{F}$ & - \\
calcGearboxLoad & velocity $\mathbb{F}$ & gearboxLoad $\mathbb{F}$ & - \\
\hline
\end{tabular}
\end{center}

\subsection{Semantics}

\subsubsection{State Variables}

None

\subsubsection{Environment Variables}

None

\subsubsection{Assumptions}

Constants are initialized from the constants module

\subsubsection{Access Routine Semantics}

\noindent calcInclineForce():
\begin{itemize}
\item output: inclineForce:= carMass*gravity*sin(inclineAngle)
\end{itemize}

\noindent calcDragForce():
\begin{itemize}
\item output: dragForce:= 0.5*airDensity*frontalArea*dragCoefficient*velocity$^2$
\end{itemize}

\noindent calcLoadTorque():
\begin{itemize}
\item output: loadTorque:= dragForce + inclineForce
\end{itemize}

\noindent calcGearboxLoad():
\begin{itemize}
\item output: gearboxLoad:= (loadTorque*wheelRadius)/gearboxRatio 
\end{itemize}

\subsubsection{Local Functions}

None

\newpage

\section{MIS of CVT Simulation Module} \label{Module_cvt_simulation} \wss{Use labels for
  cross-referencing}

\wss{You can reference SRS labels, such as R\ref{R_Inputs}.}

\wss{It is also possible to use \LaTeX for hypperlinks to external documents.}

\subsection{Module}

\wss{Short name for the module}

\subsection{Uses}

\begin{itemize}
  \item Constants Module (\ref{Module_constants})
\end{itemize}

\subsection{Syntax}

\subsubsection{Exported Constants}

\subsubsection{Exported Access Programs}

\begin{center}
\begin{tabular}{p{2cm} p{4cm} p{4cm} p{2cm}}
\hline
\textbf{Name} & \textbf{In} & \textbf{Out} & \textbf{Exceptions} \\
\hline
\wss{accessProg} & - & - & - \\
\hline
\end{tabular}
\end{center}

\subsection{Semantics}

\subsubsection{State Variables}

\wss{Not all modules will have state variables.  State variables give the module
  a memory.}

\subsubsection{Environment Variables}

\wss{This section is not necessary for all modules.  Its purpose is to capture
  when the module has external interaction with the environment, such as for a
  device driver, screen interface, keyboard, file, etc.}

\subsubsection{Assumptions}

\wss{Try to minimize assumptions and anticipate programmer errors via
  exceptions, but for practical purposes assumptions are sometimes appropriate.}

\subsubsection{Access Routine Semantics}

\noindent \wss{accessProg}():
\begin{itemize}
\item transition: \wss{if appropriate} 
\item output: \wss{if appropriate} 
\item exception: \wss{if appropriate} 
\end{itemize}

\wss{A module without environment variables or state variables is unlikely to
  have a state transition.  In this case a state transition can only occur if
  the module is changing the state of another module.}

\wss{Modules rarely have both a transition and an output.  In most cases you
  will have one or the other.}

\subsubsection{Local Functions}

\wss{As appropriate} \wss{These functions are for the purpose of specification.
  They are not necessarily something that is going to be implemented
  explicitly.  Even if they are implemented, they are not exported; they only
  have local scope.}

\newpage

\section{MIS of Input Module} \label{Module_input}

\subsection{Module}

initializer

\subsection{Uses}

None.

\subsection{Syntax}

\subsubsection{Exported Constants}
None.

\subsubsection{Exported Access Programs}

\begin{center}
\begin{tabular}{p{2cm} p{4cm} p{4cm} p{2cm}}
\hline
\textbf{Name} & \textbf{In} & \textbf{Out} & \textbf{Exceptions} \\
\hline
initialize & input ($\mathbb{I}$) & state ($\mathbb{S}$) & - \\
\hline
\end{tabular}
\end{center}

\subsection{Semantics}

\subsubsection{State Variables}

None.

\subsubsection{Environment Variables}

None.

\subsubsection{Assumptions}

None.

\subsubsection{Access Routine Semantics}

\noindent initialize(input):
\begin{itemize}
\item transition: converts input into the initial state of the simulation
\item exception: \wss{if appropriate} 
\end{itemize}

\subsubsection{Local Functions}

None.

\newpage

\section{MIS of ODE Solver Module} \label{Module_ODE_solver}

\subsection{Module}

ODE Solver

\subsection{Uses}

\begin{itemize}
  \item Constants Module (\ref{Module_constants})
  \item CVT Simulation Module (\ref{Module_cvt_simulation})
  \item External Forces Module (\ref{Module_external_forces})
  \item Engine Simulator Module (\ref{Module_engine_simulator})
  \item State Module (\ref{Module_state})
\end{itemize}

\subsection{Syntax}

\subsubsection{Exported Constants}
None

\subsubsection{Exported Access Programs}

\begin{center}
\begin{tabular}{p{2cm} p{4cm} p{4cm} p{2cm}}
\hline
\textbf{Name} & \textbf{In} & \textbf{Out} & \textbf{Exceptions} \\
\hline
solveIvp & list of enginerAngularAcceleration $\mathbb{R}$ and carAccleration $\mathbb{R}$ and carVelocity $\mathbb{S}$, timeSpan (tuple of $\mathbb{Z}$ and $\mathbb{Z}$), array of $\mathbb{S}$, array of $\mathbb{F}$  & list of States (State Module) & - \\
\hline
\end{tabular}
\end{center}

\subsection{Semantics}

\subsubsection{State Variables}

None

\subsubsection{Environment Variables}

None

\subsubsection{Assumptions}

None

\subsubsection{Access Routine Semantics}

\noindent solveIVP():
\begin{itemize}
\item output: states:= solve\_ivp(ode, initial\_state, time\_span, args)
\end{itemize}

\subsubsection{Local Functions}

None

\newpage

\section{MIS of Main Module} \label{Module_main}
\subsection{Module}

Main

\subsection{Uses}

\begin{itemize}
  \item Communication Module (\ref{Module_communication})
  \item Visualizer Module (\ref{Module_visualizer})
\end{itemize}

\subsection{Syntax}

\subsubsection{Exported Constants}

\subsubsection{Exported Access Programs}

\begin{center}
\begin{tabular}{p{2cm} p{4cm} p{4cm} p{2cm}}
\hline
\textbf{Name} & \textbf{In} & \textbf{Out} & \textbf{Exceptions} \\
\hline
main & - & - & - \\
\hline
\end{tabular}
\end{center}

\subsection{Semantics}

\subsubsection{State Variables}

None

\subsubsection{Environment Variables}

None

\subsubsection{Assumptions}


The GUI module is assumed to be running in the background and is used to display the results of the simulation.

\subsubsection{Access Routine Semantics}

\noindent main():
\begin{itemize}
\item transition: Connects the backend controller module to the visualizer module.
\end{itemize}


\subsubsection{Local Functions}

None

\newpage

\section{MIS of Playback Module} \label{Module_playback} 


\subsection{Module}

Playback

\subsection{Uses}
None.

\subsection{Syntax}

\subsubsection{Exported Constants}
carSpinTransform: Transform

\subsubsection{Exported Access Programs}

\begin{center}
\begin{tabular}{p{4cm} p{4cm} p{4cm} p{2cm}}
\hline
\textbf{Name} & \textbf{In} & \textbf{Out} & \textbf{Exceptions} \\
\hline
StartPlayback & - & - & - \\
RestartPlayback & - & - & - \\
PausePlayback & - & - & - \\
PlaybackCoroutine & - & - & - \\
\hline
\end{tabular}
\end{center}

\subsection{Semantics}

\subsubsection{State Variables}

\begin{itemize}
  \item isPlaying: $\mathbb{B}$
  \item currentIndex: $\mathbb{I}$
  \item startTime: $\mathbb{F}$ 
  \item carTransform: Transform
\end{itemize}

\subsubsection{Environment Variables}

\begin{itemize}
  \item Start Button: Button
  \item Restart Button: Button
  \item Pause Button: Button
\end{itemize}

\subsubsection{Assumptions}

Assume that there is data to playback.
\subsubsection{Access Routine Semantics}

\noindent StartPlayback():
\begin{itemize}
\item transition: isPlaying:= True, currentIndex:= 0, startTime:= time.time()
\end{itemize}

\noindent PausePlayback():
\begin{itemize}
\item transition: isPlaying:= False
\end{itemize}

\noindent RestartPlayback():
\begin{itemize}
\item transition: isPlaying:= False, currentIndex:= 0, startTime:= time.time(), carTransform:= back to start position
\end{itemize}

\noindent PlaybackCoroutine():
\begin{itemize}
\item transition: carTransform updates to new positions, carSpinTransform updates to new rotations based on position
\end{itemize}

\subsubsection{Local Functions}

None

\newpage

\section{MIS of Visualizer Module} \label{Module_visualizer} \wss{Use labels for
  cross-referencing}

\wss{You can reference SRS labels, such as R\ref{R_Inputs}.}

\wss{It is also possible to use \LaTeX for hypperlinks to external documents.}

\subsection{Module}

\wss{Short name for the module}

\subsection{Uses}

\begin{itemize}
  \item GUI Module (\ref{Module_GUI})
\end{itemize}

\subsection{Syntax}

\subsubsection{Exported Constants}

\subsubsection{Exported Access Programs}

\begin{center}
\begin{tabular}{p{2cm} p{4cm} p{4cm} p{2cm}}
\hline
\textbf{Name} & \textbf{In} & \textbf{Out} & \textbf{Exceptions} \\
\hline
\wss{accessProg} & - & - & - \\
\hline
\end{tabular}
\end{center}

\subsection{Semantics}

\subsubsection{State Variables}

\wss{Not all modules will have state variables.  State variables give the module
  a memory.}

\subsubsection{Environment Variables}

\wss{This section is not necessary for all modules.  Its purpose is to capture
  when the module has external interaction with the environment, such as for a
  device driver, screen interface, keyboard, file, etc.}

\subsubsection{Assumptions}

\wss{Try to minimize assumptions and anticipate programmer errors via
  exceptions, but for practical purposes assumptions are sometimes appropriate.}

\subsubsection{Access Routine Semantics}

\noindent \wss{accessProg}():
\begin{itemize}
\item transition: \wss{if appropriate} 
\item output: \wss{if appropriate} 
\item exception: \wss{if appropriate} 
\end{itemize}

\wss{A module without environment variables or state variables is unlikely to
  have a state transition.  In this case a state transition can only occur if
  the module is changing the state of another module.}

\wss{Modules rarely have both a transition and an output.  In most cases you
  will have one or the other.}

\subsubsection{Local Functions}

\wss{As appropriate} \wss{These functions are for the purpose of specification.
  They are not necessarily something that is going to be implemented
  explicitly.  Even if they are implemented, they are not exported; they only
  have local scope.}

\newpage

\section{MIS of Constants Module} \label{Module_constants} 
\subsection{Module}

Constants

\subsection{Uses}
None.

\subsection{Syntax}

\subsubsection{Exported Constants}

\begin{itemize}
  \item \texttt{ENGINE\_INERTIA}: A positive real value ($\mathbb{R}_{+}$) representing the inertia of the current car's engine (in $\text{kg}\cdot\text{m}^2$) used for calculations involving car specifications.
  \item \texttt{GEARBOX\_RATIO}: A positive real value ($\mathbb{R}_{+}$) representing the current car's gearbox ratio (unitless) used for calculations involving car specifications.
  \item \texttt{FRONTAL\_AREA}: A positive real value ($\mathbb{R}_{+}$) representing the current car's frontal area (in $\text{m}^2$) used for calculations involving car specifications.
  \item \texttt{DRAG\_COEFFICIENT}: A positive real value ($\mathbb{R}_{+}$) representing the current car's drag coefficient (unitless) used for calculations involving car specifications.
  \item \texttt{CAR\_WEIGHT}: A positive real value ($\mathbb{R}_{+}$) representing the current car's weight (in $\text{lbs}$) used for calculations involving car specifications.
  \item \texttt{CAR\_MASS}: A positive real value ($\mathbb{R}_{+}$) representing the current car's weight converted to kilograms (in $\text{kg}$) used for calculations involving car specifications.
  \item \texttt{WHEEL\_RADIUS}: A positive real value ($\mathbb{R}_{+}$) representing the current car's wheel radius (in $\text{m}$) used for calculations involving car specifications.
  \item \texttt{AIR\_DENSITY}: A positive real value ($\mathbb{R}_{+}$), set at \texttt{1.225} (in $\text{kg/m}^3$).
  \item \texttt{GRAVITY}: A positive real value ($\mathbb{R}_{+}$), set at \texttt{9.80665} (in $\text{m/s}^2$).
  \item \texttt{engineSpecs} A list of dictionaries representing various engine rpm's and corresponding torque values (in $\text{ft*lbs}$):=[
    {"rpm": 2400, "torque": 18.5},
    {"rpm": 2600, "torque": 18.1},
    {"rpm": 2800, "torque": 17.4},
    {"rpm": 3000, "torque": 16.6},
    {"rpm": 3200, "torque": 15.4},
    {"rpm": 3400, "torque": 14.5},
    {"rpm": 3600, "torque": 13.5}]
  \item \texttt{engineData}: A list of dictionary values for angular velocity(in $\text{rad/s}$), torque(in $\text{N*m}$), and power($\text{torque*angular velocity}$) converting the above \texttt{engineSpecs} into SI units.
  \item \texttt{angular\_velocities}: A list of angular velocity values (in rad/s) extracted from \texttt{engineData}.
  \item \texttt{torques}: A list of torque values (in N*m) extracted from \texttt{engineData}.
  \item \texttt{powers}: A list of power values (in watts) calculated from \texttt{engineData}.
  \item \texttt{torque\_curve}: A cubic interpolation function that maps \texttt{angular\_velocities} to \texttt{torques}, created using the \texttt{interp1d} method with extrapolation for values outside the range.
  
\end{itemize}


\subsubsection{Exported Access Programs}

\begin{center}
\begin{tabular}{p{2cm} p{4cm} p{4cm} p{2cm}}
\hline
\textbf{Name} & \textbf{In} & \textbf{Out} & \textbf{Exceptions} \\
\hline
constants & - & - & - \\
\hline
\end{tabular}
\end{center}

\subsection{Semantics}

\subsubsection{State Variables}

None.

\subsubsection{Environment Variables}

None.

\subsubsection{Assumptions}

None.

\subsubsection{Access Routine Semantics}

None.

\subsubsection{Local Functions}

None.

\newpage

\section{MIS of State Module} \label{Module_state} \wss{Use labels for
  cross-referencing}

\wss{You can reference SRS labels, such as R\ref{R_Inputs}.}

\wss{It is also possible to use \LaTeX for hypperlinks to external documents.}

\subsection{Module}

\wss{Short name for the module}

\subsection{Uses}
None.

\subsection{Syntax}

\subsubsection{Exported Constants}

\subsubsection{Exported Access Programs}

\begin{center}
\begin{tabular}{p{2cm} p{4cm} p{4cm} p{2cm}}
\hline
\textbf{Name} & \textbf{In} & \textbf{Out} & \textbf{Exceptions} \\
\hline
\wss{accessProg} & - & - & - \\
\hline
\end{tabular}
\end{center}

\subsection{Semantics}

\subsubsection{State Variables}

\wss{Not all modules will have state variables.  State variables give the module
  a memory.}

\subsubsection{Environment Variables}

\wss{This section is not necessary for all modules.  Its purpose is to capture
  when the module has external interaction with the environment, such as for a
  device driver, screen interface, keyboard, file, etc.}

\subsubsection{Assumptions}

\wss{Try to minimize assumptions and anticipate programmer errors via
  exceptions, but for practical purposes assumptions are sometimes appropriate.}

\subsubsection{Access Routine Semantics}

\noindent \wss{accessProg}():
\begin{itemize}
\item transition: \wss{if appropriate} 
\item output: \wss{if appropriate} 
\item exception: \wss{if appropriate} 
\end{itemize}

\wss{A module without environment variables or state variables is unlikely to
  have a state transition.  In this case a state transition can only occur if
  the module is changing the state of another module.}

\wss{Modules rarely have both a transition and an output.  In most cases you
  will have one or the other.}

\subsubsection{Local Functions}

\wss{As appropriate} \wss{These functions are for the purpose of specification.
  They are not necessarily something that is going to be implemented
  explicitly.  Even if they are implemented, they are not exported; they only
  have local scope.}

\newpage

\section{MIS of Backend Controller Module} \label{Module_backend_controller} \wss{Use labels for
  cross-referencing}

\wss{You can reference SRS labels, such as R\ref{R_Inputs}.}

\wss{It is also possible to use \LaTeX for hypperlinks to external documents.}

\subsection{Module}

\wss{Short name for the module}

\subsection{Uses}

\begin{itemize}
  \item Input Module (\ref{Module_input})
  \item ODE Solver Module (\ref{Module_ODE_solver})
\end{itemize}

\subsection{Syntax}

\subsubsection{Exported Constants}

\subsubsection{Exported Access Programs}

\begin{center}
\begin{tabular}{p{2cm} p{4cm} p{4cm} p{2cm}}
\hline
\textbf{Name} & \textbf{In} & \textbf{Out} & \textbf{Exceptions} \\
\hline
\wss{accessProg} & - & - & - \\
\hline
\end{tabular}
\end{center}

\subsection{Semantics}

\subsubsection{State Variables}

\wss{Not all modules will have state variables.  State variables give the module
  a memory.}

\subsubsection{Environment Variables}

\wss{This section is not necessary for all modules.  Its purpose is to capture
  when the module has external interaction with the environment, such as for a
  device driver, screen interface, keyboard, file, etc.}

\subsubsection{Assumptions}

\wss{Try to minimize assumptions and anticipate programmer errors via
  exceptions, but for practical purposes assumptions are sometimes appropriate.}

\subsubsection{Access Routine Semantics}

\noindent \wss{accessProg}():
\begin{itemize}
\item transition: \wss{if appropriate} 
\item output: \wss{if appropriate} 
\item exception: \wss{if appropriate} 
\end{itemize}

\wss{A module without environment variables or state variables is unlikely to
  have a state transition.  In this case a state transition can only occur if
  the module is changing the state of another module.}

\wss{Modules rarely have both a transition and an output.  In most cases you
  will have one or the other.}

\subsubsection{Local Functions}

\wss{As appropriate} \wss{These functions are for the purpose of specification.
  They are not necessarily something that is going to be implemented
  explicitly.  Even if they are implemented, they are not exported; they only
  have local scope.}

\newpage

\section{MIS of GUI Module} \label{Module_GUI}

\subsection{Module}

\texttt{gui}

\subsection{Uses}

None.

\subsection{Syntax}

\subsubsection{Exported Constants}
None.

\subsubsection{Exported Access Programs}

\begin{center}
\begin{tabular}{p{2cm} p{4cm} p{4cm} p{2cm}}
\hline
\textbf{Name} & \textbf{In} & \textbf{Out} & \textbf{Exceptions} \\
\hline
gui & None & None & - \\
\hline
\end{tabular}
\end{center}

\subsection{Semantics}

\subsubsection{State Variables}

\begin{itemize}
  \item Button states (Boolean for clicked state)
  \item Input Fields ($\mathbb{I}$)
  
\end{itemize}

\subsubsection{Environment Variables}

\begin{itemize}
  \item Keyboard ($\mathbf{Z}_{+}$ for keycodes describing the key pressed)
  \item Mouse (Boolean for click state and  $\mathbf{Z}_{+}$ for cursor position)
  \item Screen ($\mathbf{Z}_{+}$ for width and height in pixels)
\end{itemize}

\subsubsection{Assumptions}

None.

\subsubsection{Access Routine Semantics}

\noindent \texttt{gui}():
\begin{itemize}
\item transition: Provides methods from Unity to build and deploy a GUI to the Visualizer Module \ref{Module_visualizer}
\end{itemize}

\subsubsection{Local Functions}

None.

\newpage

\section{MIS of File Output Module} \label{Module_file_output}

\subsection{Module}

output

\subsection{Uses}
None.

\subsection{Syntax}

\subsubsection{Exported Constants}
None.

\subsubsection{Exported Access Programs}

\begin{center}
\begin{tabular}{p{2cm} p{4cm} p{4cm} p{2cm}}
\hline
\textbf{Name} & \textbf{In} & \textbf{Out} & \textbf{Exceptions} \\
\hline
write & outputPath (String) & - & - \\
\hline
\end{tabular}
\end{center}

\subsection{Semantics}

\subsubsection{State Variables}

\begin{itemize}
  \item states: $\mathbb{S}^n$, where each entry represents the state of the car at a given time.
\end{itemize}

\subsubsection{Environment Variables}

None.

\subsubsection{Assumptions}

The file path given can be written to.

\subsubsection{Access Routine Semantics}

\noindent write(outputPath):
\begin{itemize}
\item output: Writes the states to a file at the given path.
\item exception: \wss{if appropriate} 
\end{itemize}

\subsubsection{Local Functions}

None.

\newpage

\section{MIS of Communication Module} \label{Module_communication}

\subsection{Module}

communication

\subsection{Uses}

\begin{itemize}
  \item Backend Controller Module (\ref{Module_backend_controller})
\end{itemize}

\subsection{Syntax}

\subsubsection{Exported Constants}
None.

\subsubsection{Exported Access Programs}

\begin{center}
\begin{tabular}{p{2cm} p{4cm} p{4cm} p{2cm}}
\hline
\textbf{Name} & \textbf{In} & \textbf{Out} & \textbf{Exceptions} \\
\hline
frontToBack & input ($\mathbb{I}$) & ouput ($\mathbb{I}$) & - \\
backToFront & - & states ($\mathbb{S}^n$) & - \\
\hline
\end{tabular}
\end{center}

\subsection{Semantics}

\subsubsection{State Variables}

\begin{itemize}
  \item mainPath: a String representing the path to the main file.
  \item outputPath: a String representing the path to the file to be read.
\end{itemize}

\subsubsection{Environment Variables}

\begin{itemize}
  \item pythonPath: a String representing the path to the python environment.
\end{itemize}

\subsubsection{Assumptions}

All files are in the correct location matching the given paths.

\subsubsection{Access Routine Semantics}

\noindent frontToBack(input):
\begin{itemize}
\item transition: Sends the given parameters to the backend controller.
\item exception: \wss{if appropriate} 
\end{itemize}

\noindent backToFront():
\begin{itemize}
  \item transition: Reads the states from the output file.
  \item exception: \wss{if appropriate}
\end{itemize}

\subsubsection{Local Functions}

None.

\newpage

\bibliographystyle {plainnat}
\bibliography {../../../refs/References}

\newpage

\section{Appendix} \label{Appendix}

\wss{Extra information if required}

\newpage{}

\section*{Appendix --- Reflection}

\wss{Not required for CAS 741 projects}

The information in this section will be used to evaluate the team members on the
graduate attribute of Problem Analysis and Design.

\input{../../Reflection.tex}

\begin{enumerate}
  \item What went well while writing this deliverable? 
  \item What pain points did you experience during this deliverable, and how
    did you resolve them?
  \item Which of your design decisions stemmed from speaking to your client(s)
  or a proxy (e.g. your peers, stakeholders, potential users)? For those that
  were not, why, and where did they come from?
  \item While creating the design doc, what parts of your other documents (e.g.
  requirements, hazard analysis, etc), it any, needed to be changed, and why?
  \item What are the limitations of your solution?  Put another way, given
  unlimited resources, what could you do to make the project better? (LO\_ProbSolutions)
  \item Give a brief overview of other design solutions you considered.  What
  are the benefits and tradeoffs of those other designs compared with the chosen
  design?  From all the potential options, why did you select the documented design?
  (LO\_Explores)
\end{enumerate}


\end{document}