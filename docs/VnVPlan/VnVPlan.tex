\documentclass[12pt, titlepage]{article}

\usepackage{float}
\usepackage{geometry}
\usepackage{booktabs}
\usepackage{tabularx}
\usepackage{hyperref}
\hypersetup{
    colorlinks,
    citecolor=blue,
    filecolor=black,
    linkcolor=red,
    urlcolor=blue
}
\usepackage[round]{natbib}
\usepackage{amsmath, mathtools}

\input{../Comments}
%% Common Parts

\newcommand{\progname}{Baja Dynamics} % PUT YOUR PROGRAM NAME HERE
\newcommand{\authname}{Team \#17, Team Name
\\ Grace McKenna
\\ Travis Wing
\\ Cameron Dunn
\\ Kai Arseneau} % AUTHOR NAMES                   

\usepackage{hyperref}
    \hypersetup{colorlinks=true, linkcolor=blue, citecolor=blue, filecolor=blue,
                urlcolor=blue, unicode=false}
    \urlstyle{same}
                                


\begin{document}

\title{System Verification and Validation Plan for \progname{}} 
\author{\authname}
\date{\today}
	
\maketitle

\pagenumbering{roman}

\section*{Revision History}

\begin{tabularx}{\textwidth}{p{3cm}p{2cm}X}
\toprule {\bf Date} & {\bf Version} & {\bf Notes}\\
\midrule
Date 1 & 1.0 & Notes\\
Date 2 & 1.1 & Notes\\
\bottomrule
\end{tabularx}

~\\
\wss{The intention of the VnV plan is to increase confidence in the software.
However, this does not mean listing every verification and validation technique
that has ever been devised.  The VnV plan should also be a \textbf{feasible}
plan. Execution of the plan should be possible with the time and team available.
If the full plan cannot be completed during the time available, it can either be
modified to ``fake it'', or a better solution is to add a section describing
what work has been completed and what work is still planned for the future.}

\wss{The VnV plan is typically started after the requirements stage, but before
the design stage.  This means that the sections related to unit testing cannot
initially be completed.  The sections will be filled in after the design stage
is complete.  the final version of the VnV plan should have all sections filled
in.}

\newpage

\tableofcontents

\listoftables
\wss{Remove this section if it isn't needed}

\listoffigures
\wss{Remove this section if it isn't needed}

\newpage

\section{Symbols, Abbreviations, and Acronyms}

\begin{table}[h] % Optional [h] to place table here
  \raggedright
  \begin{tabular}{l l} 
    \toprule		
    \textbf{acronym} & \textbf{definition}\\
    \midrule
    CD & Continuous Development\\
    CI & Continuous Integration\\ 
    CVT & Continuous Variable Transmission\\
    FR & Functional Requirement\\
    GPS & Global Positioning System\\
    GUI & Graphical User Interface\\
    IM & Instance Model\\
    MG & Module Guide\\
    MIS & Module Interface Specification\\
    MSE & Mean Squared Error\\
    NFR & Nonfunctional Requirement\\
    PR & Pull Request\\
    RPM & Revolutions Per Minute\\
    SRS & Software Requirements Specification\\
    VnV & Verification and Validation\\
    \bottomrule
  \end{tabular}
  \caption{Verification and Validation Acronyms}
  \label{tab:vnv_acronyms}
\end{table}

\begin{table}[h] % Optional [h] to place table here
  \raggedright
  \begin{tabular}{l l} 
    \toprule		
    \textbf{constant} & \textbf{value}\\
    \midrule
    $m_{\text{max}}$ & $400 kg$\\
    \bottomrule
  \end{tabular}
  \caption{Verification and Validation Constants}
  \label{tab:vnv_constants}
\end{table}

\wss{symbols, abbreviations, or acronyms --- you can simply reference the SRS
  \citep{SRS} tables, if appropriate}

\wss{Remove this section if it isn't needed}

\newpage

\pagenumbering{arabic}

This document ... \wss{provide an introductory blurb and roadmap of the
  Verification and Validation plan}

\noindent This document serves as a guide for how we plan to verify and validate the \progname{} system.
Within is a detailed plan of all the tests that will be conducted to ensure that the system is functioning as intended.
The document is divided into several sections going from the general plan to the specific tests that will be conducted.
On the verification side, the document lists all all the tests that will be conducted to verify the performance of the system.
The tests are divided into functional and nonfunctional requirements and then separate system and unit test for each.
There are also multiple checklist in place to verify that all the requirements both within and outside the document are met.
On the validation side, the document outlines all the plans in place to validate the results of the system.
This includes how we plan to compare our results against real world data and how we plan to validate the usability of the system.
Overall, this document comprehensively outlines all the plans in place to verify the system and validate its results.

\section{General Information}

This section serves as an overall summary of the verification and validation plan.
It includes a summary of the project and the overall objectives of the VnV plan.
The relevant documentation and other specifications are also listed.
In summary, this section provides a high level overview of the VnV plan.

\subsection{Summary}

The software that will be tested is the \progname{}, this is a system that simulates the Continuously Variable Transmission (CVT) system of a Baja vehicle.
The system has three components, a user interface which will allow the user to input the parameters of the CVT system and visualize the output of the simulation.
A mathematical component that will take the input parameters and simulate the CVT system.
There is also a 3D model component that will display the some models of the CVT system. 
Each of these components covered will be verified and validated to ensure that the system is functioning as intended.

\subsection{Objectives}

\wss{State what is intended to be accomplished.  The objective will be around
  the qualities that are most important for your project.  You might have
  something like: ``build confidence in the software correctness,''
  ``demonstrate adequate usability.'' etc.  You won't list all of the qualities,
  just those that are most important.}

\wss{You should also list the objectives that are out of scope.  You don't have 
the resources to do everything, so what will you be leaving out.  For instance, 
if you are not going to verify the quality of usability, state this.  It is also 
worthwhile to justify why the objectives are left out.}

\wss{The objectives are important because they highlight that you are aware of 
limitations in your resources for verification and validation.  You can't do everything, 
so what are you going to prioritize?  As an example, if your system depends on an 
external library, you can explicitly state that you will assume that external library 
has already been verified by its implementation team.}

\noindent The objectives of testing our CVT system are to ensure the accuracy and reliability of our outputs. 
This includes evaluating that our mathematical model provides correct outputs based on provided input parameters specifically looking at the calculations of RPM, velocity, acceleration, distance, torque of the engine, clamping forces, CVT ratio, Torque and Belt slippage. 
We aim to emphasize the description over specification to make the CVT's system's behavior understandable by demonstrating how input changes influence the outputs.
We will verify that our 3D model and user interface allow users to easily input parameters to the system and interpret meaningful results. 
Our validation tests will focus on graphical models based on current available data to describe how the outputs should behave. 
We will aim to quantify these results by analyzing the difference between expected versus actual output. 
By referencing our outputs to real world data we can validate the accuracy of our mathematical model.
We will conduct useability testing to ensure that the interface is intuitive and allows users to easily input parameters, visualize results and understand the performance of the CVT.
\\
\noindent For our project attempting to optimize our systems outputs will be considered out of scope. 
Additionally, we will not verify any third party libraries used in our software as we assume that this has been previously validated. 

\subsection{Challenge Level and Extras}

The challenge level of this project is general.\\
\newline
The extras for this project are the following:\\
\textbf{Validation Report}
{\begin{itemize}
  \item The system will undergo validation testing to verify that accuracy of the simulation.
  \item The validation report will be included in the final documentation.
\end{itemize}}
{\noindent}
\textbf{Usability Testing}
{\begin{itemize}
  \item The system will undergo usability testing by members of the McMaster Baja team to ensure that
  the user interface is intuitive and easy to use.
  \item We will compile the results of the usability testing into a report that will be included in the final documentation.
\end{itemize}}

\subsection{Relevant Documentation}

\wss{Reference relevant documentation.  This will definitely include your SRS
  and your other project documents (design documents, like MG, MIS, etc).  You
  can include these even before they are written, since by the time the project
  is done, they will be written.  You can create BibTeX entries for your
  documents and within those entries include a hyperlink to the documents.}

\noindent \textbf{Software Requirements Specification} \citet{SRS}
\newline
This document is relevant because it it list all the requirements that the system should have and specifies how they should be achieved.
Requirements serve an important role in the verification process as the metric for determining if the system is functioning as intended.
When we do the verification of the system we will compare the system's capabilities and outputs to the requirements from the SRS document.
This will tell us what parts of the system are working as intended and which need to be changed or updated.
For validation, the specification serves and equivalent role as the metric for determining if the system is implemented correctly.
The specifications will be compared with the implementations of the system to see if they correspond.
If they do not, then the system will need to be updated so the the implementations match with the specifications.
Overall, the SRS document is important for verification and validation as it provides the metrics for determining their success.
\bigskip
\newline
\textbf{Module Guide}
\newline
The Module Guide will serve as a reference for the components of the system.
Each module listed in the MG will have a corresponding test case in the VnV plan.
\bigskip
\newline
\textbf{Module Interface Specification}
\newline
The Module Interface Specification serves as a reference for the functions and methods of the system.
Each function and method listed in the MIS will have a corresponding test case in the VnV plan.
\bigskip
\newline
\textbf{Verification and Validation Report}
\newline
The Verification and Validation Report will be referenced to ensure that all the components of the system have been verified and validated.

\wss{Don't just list the other documents.  You should explain why they are relevant and 
how they relate to your VnV efforts.}

\section{Plan}

\wss{Introduce this section.  You can provide a roadmap of the sections to
  come.}

This section outlines the overall plans for the verification and validation of the system.
This includes our approaches, techniques and tools that will be used.
Our approaches our outlined and each person's role is specified.
There is also a plan listed for both the verification and validation along with an implementation specification.
Any tools that we plan to use to automate the process are also listed in this section.
Overall this section details how we plan to verify and validate the system.

\subsection{Verification and Validation Team}

\wss{Your teammates.  Maybe your supervisor.
  You should do more than list names.  You should say what each person's role is
  for the project's verification.  A table is a good way to summarize this information.}

\newgeometry{left=0.5cm, right=0.5cm, top=2cm, bottom=2cm}
\begin{table}[h!]
  \centering
  \begin{tabular}{|p{3cm}|p{3cm}|p{10cm}|}
  \hline
  \textbf{Name} & \textbf{Role} & \textbf{Responsibilities} \\
  \hline
  Kai Arseneau & Data Verification Lead & Analyzes data outputs to ensure trends align with expected performance and validates consistency across simulation runs. \\
  \hline
  Travis Wing & GUI Verification Lead & Ensures user interface functionality and usability, verifying that GUI elements accurately reflect simulation data and respond as intended. \\
  \hline
  Cameron Dunn & Experimental Data Comparison Lead & Compares simulation outputs to real-world data and scripts, verifying the simulation's accuracy and adjusting model parameters as needed. \\
  \hline
  Grace McKenna & Quality Assurance Lead & Conducts comprehensive testing of simulation processes and components, identifying and documenting issues, and ensuring overall software reliability. \\
  \hline
  Ariel Wolle & Team Lead & Supports the project by overseeing resources and approving initiatives that facilitate enhanced data acquisition or other verification efforts. \\
  \hline
  Daksh Mathur &  Data Acquisition Lead & Acts as the primary contact for real-world data acquisition, providing essential data for validation and supporting data needs beyond the project group. \\
  \hline
  Benjamin Waldie & Mechanical Advisor & Collaborates with the team to validate mathematical models within the simulation, ensuring mechanical accuracy and guiding technical correctness. \\
  \hline
  Duncan Shearer and Leonardo Ansari & Primary User & Utilizes the simulation tool to tune CVTs, providing feedback on tool usability and functionality to ensure it meets real-world tuning needs. \\
  \hline
  Dr. Spencer Smith & Project Advisor / Supervisor & Reviews and provides final feedback, ensuring verification meets project standards and advising on methodology improvements. \\
  \hline
  \end{tabular}
  \caption{Verification and Validation Team for CVT Simulation Project}
  \label{tab:vnv_team}
\end{table}
\restoregeometry
    

\subsection{SRS Verification Plan}

\wss{List any approaches you intend to use for SRS verification.  This may
  include ad hoc feedback from reviewers, like your classmates (like your
  primary reviewer), or you may plan for something more rigorous/systematic.}

\wss{If you have a supervisor for the project, you shouldn't just say they will
read over the SRS.  You should explain your structured approach to the review.
Will you have a meeting?  What will you present?  What questions will you ask?
Will you give them instructions for a task-based inspection?  Will you use your
issue tracker?}

\wss{Maybe create an SRS checklist?}

\noindent To ensure that the SRS document is accurate and complete, we will be conducting a series of reviews.
These reviews will be conducted by the team, our supervisor, and our stakeholders.
The team will use the below checklist to verify the quality of the SRS document:
\begin{itemize}
  \item Review the SRS document for spelling, grammatical and formatting errors
  \item Check the applicability of each theoretical models
  \item Check the usefulness of each data definitions
  \item Check the validity of each instance models
\end{itemize}

\subsubsection*{Review Meetings}
We will conduct meetings with experienced team members and alumni to review the SRS document.
These meetings will focus on the theoretical models, data definitions, and instance models.
We will also conduct a practicality check to ensure that all requirements are feasible and measurable.
Finally the assumptions will also be verified to ensure that they won't negatively impact the accuracy of the simulation.

\subsubsection*{Baja Team Review}
Talk here about sharing the document to the team as a whole, include a checklist and receive issues on github.

We will share the SRS document with the entire McMaster Baja Racing team for review.
This will allow us to get feedback from a wide range of people with different backgrounds and experiences.
The team will use the checklist provided above as a guide for their reviews.
Afterwards we will compile all of their feedback and address any issues that they have raised.

\subsubsection*{Capstone Team Testing}

We will conduct unit tests on the mathematical models provided by the SRS document.
These will be done on their implementations to ensure that they are functioning as intended.
If any issues are found, we will reference the assumptions and requirements in the SRS document to determine the cause of the issue.
Additionally, the traceability matrix will be used to ensure that the mapping between requirements and goals is logical. 

\subsection{Design Verification Plan}

\wss{Plans for design verification}

\wss{The review will include reviews by your classmates}

\wss{Create a checklists?}

\noindent To ensure that the CVT system is functional and accurate, a design verification plan will be conducted on the system. 
The components of the system that will be tested are the graphical user interface, data output and visualization and the mathematical model. 
The design verification plan will include, reviews from fellow classmates, stakeholder reviews and specification testing based on the functional and nonfunctional requirements of the system. 
The system will receive Stakeholder review providing us with relevant feedback on the functionality and interface of our design.
The system will undergo reviews from fellow classmates, providing our team with quality feedback from an outside perspective. 
A checklist has been designed below to which fellow classmates and stakeholders can use to verify the system.  


\begin{enumerate}
  \item For each functional requirement there is a at least one corresponding system test.
  \item For each non functional requirement there is a at least one corresponding system test.  
  \item The system tests contain tests for boundary cases for user inputs, empty inputs and edge cases. 
  \item Each instance model (IM) within the SRS document shall correspond to a unit test. 
  \item Each function written in our mathematical python model shall correspond to at least one unit test. 
  \item There exists system tests for each component: user interface, mathematical model, data visualization and 3D model. 
\end{enumerate}
\subsection{Verification and Validation Plan Verification Plan}

\wss{The verification and validation plan is an artifact that should also be
verified.  Techniques for this include review and mutation testing.}

\wss{The review will include reviews by your classmates}

\wss{Create a checklists?}

To ensure that the Verification and Validation Plan is sufficient, this document will undergo a verification plan.
This plan will review the contents of the document to make sure that all necessary components are included.
It will be reviewed by our team, supervisor, stakeholders and classmates.
Below is a checklist to use as reference for the verification of the Verification and Validation Plan.
\bigskip
\newline
\textbf{Overall}
\begin{itemize}
  \item The document is well organized and easy to follow
  \item The document is free of spelling, grammatical and formatting errors
  \item The document is consistent with all other documentation and implementation
  \item The document is complete and has all components filled out
  \item References are included and correct
\end{itemize}
\noindent
\textbf{Section 1}
\begin{itemize}
  \item Tables included are relevant and necessary
  \item Tables include are correct and formatted properly
  \item Introduction sufficiently introduces the document
\end{itemize}
\noindent
\textbf{Section 2}
\begin{itemize}
  \item Roadmap effective summarizes the section
  \item Summary effectively summarizes the document
  \item Objectives are clear and concise
  \item Challenge Level correct and Extras are appropriate
  \item Relevant Documentation is matching and descriptive
\end{itemize}
\noindent
\textbf{Section 3}
\begin{itemize}
  \item Roadmap effective summarizes the section
  \item Team roles allocated properly and effectively
  \item SRS Verification Plan includes all necessary components in the checklist
  \item Design Verification Plan includes all necessary components in the checklist
  \item VnV Plan Verification Plan includes all necessary components in the checklist
  \item Implementation Verification Plan includes all techniques that are used
  \item Automated Testing and Verification Tools includes all tools that are used
  \item Software Validation Plan includes all external data that is used
\end{itemize}
\noindent
\textbf{Section 4}
\begin{itemize}
  \item Roadmap effective summarizes the section
  \item All functional requirements have corresponding system tests
  \item All nonfunctional requirements have corresponding system tests
  \item Traceability table between test cases and requirements is clear
\end{itemize}
\noindent
\textbf{Section 5}
\begin{itemize}
  \item Roadmap effective summarizes the section
  \item Unit Testing Scope is correctly defined
  \item All functional requirements have corresponding unit tests
  \item All nonfunctional requirements have corresponding unit tests
  \item Unit tests are properly grouped into their respective modules
  \item Traceability table between test cases and modules is clear and includes all modules
\end{itemize}
\noindent
\textbf{Section 6}
\begin{itemize}
  \item Roadmap effective summarizes the section and includes all additions
  \item Additional information is relevant and necessary
  \item Included sections are informative and properly filled out
\end{itemize}


\subsection{Implementation Verification Plan}

\wss{You should at least point to the tests listed in this document and the unit
  testing plan.}

\wss{In this section you would also give any details of any plans for static
  verification of the implementation.  Potential techniques include code
  walkthroughs, code inspection, static analyzers, etc.}

\wss{The final class presentation in CAS 741 could be used as a code
walkthrough.  There is also a possibility of using the final presentation (in
CAS741) for a partial usability survey.}

For the implementation verification plan, we will be using many techniques to ensure that the system is functioning as intended.
We will use the unit tests listed in the this document as well as other common testing techniques.
It will be required that all unit tests are run and pass for any PRs to be merged into their parent branch.
We will be doing code inspections during development by ensuring that all code is reviewed by at least one other team member before merging.
Static analyzers will also be used throughout development to ensure a high quality codebase and will be integrated into our CI/CD pipeline.
Before major milestones, full code walkthroughs will be conducted which is enforced by all team members needing to approve merges into main.

\subsection{Automated Testing and Verification Tools}

\wss{What tools are you using for automated testing.  Likely a unit testing
  framework and maybe a profiling tool, like ValGrind.  Other possible tools
  include a static analyzer, make, continuous integration tools, test coverage
  tools, etc.  Explain your plans for summarizing code coverage metrics.
  Linters are another important class of tools.  For the programming language
  you select, you should look at the available linters.  There may also be tools
  that verify that coding standards have been respected, like flake9 for
  Python.}

  \noindent We will be using a wide variety of tools for automated testing and verification.
  We will use separate tools for our Unity C\# frontend and python backend.
  \bigskip
  \newline
  For the python backend we will be using the following tools:
  \begin{itemize}
    \item [\textbf{flake8}] - A Python linter that checks for PEP8 compliance.
    \item [\textbf{black}] - A Python code formatter that will ensure consistent code style.
    \item [\textbf{unittest}] - Python's built-in testing framework that will be used for unit testing.
    \item [\textbf{coverage}] - A testing framework for Python that will be used for code coverage.
  \end{itemize}

  \bigskip
  \noindent For the Unity C\# frontend we will be using the following tools:
  \begin{itemize}
    \item [\textbf{SonarLint}] - C\# linter that checks for code quality and security vulnerabilities.
    \item [\textbf{StyleCop}] - C\# linter that checks for code style and formatting.
    \item [\textbf{UTF}] - A testing framework for C\# that will be used for unit testing.
    \item [\textbf{UTR}] - A testing framework for Unity that will be used for unit testing and code coverage.
  \end{itemize}

  \noindent Additionally, we will be using GitHub Actions to automate our testing process.
  This will be done to ensure CI/CD and that our tests are run automatically on every push to the repository.

\wss{If you have already done this in the development plan, you can point to
that document.}

\wss{The details of this section will likely evolve as you get closer to the
  implementation.}

\subsection{Software Validation Plan}

\wss{If there is any external data that can be used for validation, you should
  point to it here.  If there are no plans for validation, you should state that
  here.}

\wss{You might want to use review sessions with the stakeholder to check that
the requirements document captures the right requirements.  Maybe task based
inspection?}

\wss{For those capstone teams with an external supervisor, the Rev 0 demo should 
be used as an opportunity to validate the requirements.  You should plan on 
demonstrating your project to your supervisor shortly after the scheduled Rev 0 demo.  
The feedback from your supervisor will be very useful for improving your project.}

\wss{For teams without an external supervisor, user testing can serve the same purpose 
as a Rev 0 demo for the supervisor.}

\wss{This section might reference back to the SRS verification section.}

\noindent We plan on using data collected from the McMaster Baja team to validate the accuracy of our simulation.
Some of the values we are able to collect include:
\begin{itemize}
  \item Primary RPM
  \item Secondary RPM
  \item CVT Ratio (from Primary and Secondary RPM)
  \item Wheel Speed
  \item GPS Speed
  \item GPS Position
  \item Acceleration
\end{itemize}

\noindent To validate this data against our simulation we plan on creating a script to compare them.
This script will compute the MSE between the two datasets and also plot that error over time.
Plotting the error will allow us to see in what scenarios the simulation is least accurate.
This will allow us to better diagnose the issues and allocate our resources to fix them.

\section{System Tests}

\wss{There should be text between all headings, even if it is just a roadmap of
the contents of the subsections.}

Found below are all of the system tests that will be conducted to verify the system.
These tests are split based on if they are for functional or nonfunctional requirements.

\subsection{Tests for Functional Requirements}

\wss{Subsets of the tests may be in related, so this section is divided into
  different areas.  If there are no identifiable subsets for the tests, this
  level of document structure can be removed.}

\wss{Include a blurb here to explain why the subsections below
  cover the requirements.  References to the SRS would be good here.}

This section outlines all the system tests that will be conducted to verify the requirements of the system.
Each set of tests relate to one requirement from the SRS document.

\subsubsection*{Standard Simulation Inputs}
\label{sec:standard_inputs}
The standard simulation inputs for the CVT system are as follows.
\begin{enumerate}
  \item Primary Pulley System:
  \begin{itemize}
    \item Flyweight
    \item Ramp Geometry
    \item Spring rate
    \item Spring pretension
  \end{itemize}
  \item Secondary Pulley System:
  \begin{itemize}
    \item Helix Geometry
    \item Torsional spring rate
    \item Compressional spring rate
    \item Spring pretension
  \end{itemize}
  \item Vehicle Characteristics:
  \begin{itemize}
    \item Car weight
    \item Driver weight
    \item Traction
    \item Angle of incline
  \end{itemize}
\end{enumerate}

\subsubsection*{Standard Simulation State}
\label{sec:standard_state}
The standard simulation state for the CVT system is as follows.
\begin{itemize}
  \item Vehicle is stationary.
  \item The engine's angular velocity is at $\omega_\text{idle}$.
  \item The CVT system is in a neutral state, unshifted.
  \item The belt is stationary.
\end{itemize}

\subsubsection{Vehicle Dynamics}

\wss{It would be nice to have a blurb here to explain why the subsections below
  cover the requirements.  References to the SRS would be good here.  If a section
  covers tests for input constraints, you should reference the data constraints
  table in the SRS.}

This section provides tests to verify the vehicle dynamics of the CVT system.
They are based on the following functional requirements from the SRS document.
Listed is each requirement and how the tests verify each one.
\begin{itemize}
  \item [R1:] Verifies the mathematical modeling of the CVT system
  \item [R2:] Verifies the acceleration of the vehicle as a function of time
  \item [R3:] Verifies the velocity of the vehicle as a function of time
  \item [R4:] Verifies the position of the vehicle as a function of time
\end{itemize}

\paragraph{Position}

\begin{enumerate}
  
  \item{test-1\\}
  
  Control: Manual
            
  Initial State: The state values from Section~\ref{sec:standard_state}.
  
  Input: The input values from Section~\ref{sec:standard_inputs}.
            
  Output: The resulting graph should show a position that is a monotonically increasing function over time, starting from 0. There is no upper bound. A typical output graph will look similar to the one shown in Figure~\ref{fig:position_graph}.
  
  Test Case Derivation: As long as the vehicle is able to overcome the initial force of the slope, the vehicle should continue to move forward indefinitely.
  
  How test will be performed: The script will run with the provided inputs. After execution is complete, the output position over time will be visualized as a graph which should show a smooth upward trend as time progresses. Its slope should increase over time as well. Visual inspection will follow against Figure~\ref{fig:position_graph} for confirmation of expected behaviour.
  
  \item{test-2\\}

  Control: Manual

  Initial State: The state values from Section~\ref{sec:standard_state}.
  
  Input: Provided experimental data along with the input values from Section~\ref{sec:standard_inputs}, matching those of the vehicle during experimental data collection. Sample experimental data is located here: \href{https://github.com/gr812b/CVT-Simulator/experimental-data/GPS%20LATITUDE.csv}{Latitude.csv}, \href{https://github.com/gr812b/CVT-Simulator/experimental-data/GPS%20LONGITUDE.csv}{Longitude.csv}.

  Output: Mean squared error between the simulated position and the experimental position.

  Test Case Derivation: Data from the simulation should closely match the experimental data collected from the physical car.

  How test will be performed: The script will run with the provided inputs. After execution is complete, the output of the simulation will be compared to experimental data from the physical car to ensure that the simulation is accurate. The experimental data will be configured to show the absolute distance travel since the start of the experiment.

\end{enumerate}
		
\paragraph{Velocity}

\begin{enumerate}

\item{test-1\\}

Control: Manual
					
Initial State: The state values from Section~\ref{sec:standard_state}.

Input: The input values from Section~\ref{sec:standard_inputs}.
					
Output: The resulting graph should show a velocity that is a monotonically increasing function over time, starting from zero and remaining below $v_\text{max}$, which reflects the vehicle’s physical limits. A typical output graph will look similar to the one shown in Figure~\ref{fig:velocity_graph}.

Test Case Derivation: Based on assumption REF HERE that the vehicle should not move backward under normal conditions.
Derived from the vehicle’s geometry and the CVT configuration, limiting the velocity to a defined top speed based on the system’s constraints.


How test will be performed: The script will run with the provided inputs. After execution is complete, the output velocity over time will be visualized as a graph which should show a smooth upward trend that begins at zero and approaches $v_\text{max}$ as time progresses. Visual inspection will follow against Figure~\ref{fig:velocity_graph} for confirmation of expected behaviour.
					
\item{test-2\\}

Control: Manual

Initial State: The state values from Section~\ref{sec:standard_state}.

Input: Provided experimental data along with the input values from Section~\ref{sec:standard_inputs}, matching those of the vehicle during experimental data collection. Sample experimental data is located here: \href{https://github.com/gr812b/CVT-Simulator/experimental-data/GPS%20SPEED.csv}{Speed.csv}.

Output: Mean squared error between the simulated velocity and the experimental velocity.

Test Case Derivation: Data from the simulation should closely match the experimental data collected from the physical car.

How test will be performed: The script will run with the provided inputs. After execution is complete, the output of the simulation will be compared to experimental data from the physical car to ensure that the simulation is accurate.

\end{enumerate}

\paragraph{Acceleration}

\begin{enumerate}

  \item{test-1\\}
  
  Control: Manual
            
  Initial State: The state values from Section~\ref{sec:standard_state}.
  
  Input: The input values from Section~\ref{sec:standard_inputs}.
            
  Output: The resulting graph should show an acceleration that is a smooth monotonically decreasing function over time, starting from $a_\text{max}$ and remaining above $a_\text{min}$, which reflects the engine's limits along with the maximum and minimum transmission ratio. A typical output graph will look similar to the one shown in Figure~\ref{fig:acceleration_graph}.
  
  Test Case Derivation: Based on the engine's output limits, going no lower than $\omega_\text{idle}$ and no higher than $\omega_\text{max}$.
  Based on air resistance, which increases with the square of velocity, alongside the diminishing torque output of the engine.
  
  How test will be performed: The script will run with the provided inputs. After execution is complete, the output will be visualized as a graph. Visual inspection will follow against Figure~\ref{fig:acceleration_graph} for confirmation of expected behaviour.

  \item{test-2\\}

  Control: Manual

  Initial State: The state values from Section~\ref{sec:standard_state}.

  Input: Provided experimental data along with the input values from Section~\ref{sec:standard_inputs}, matching those of the vehicle during experimental data collection. Sample experimental data is located here: \href{https://github.com/gr812b/CVT-Simulator/experimental-data/IMU%20ACCEL%20X.csv}{Accel X.csv}, \href{https://github.com/gr812b/CVT-Simulator/experimental-data/IMU%20ACCEL%20Y.csv}{Accel Y.csv}, \href{https://github.com/gr812b/CVT-Simulator/experimental-data/IMU%20ACCEL%20Z.csv}{Accel Z.csv}.

  Output: Mean squared error between the simulated acceleration and the experimental acceleration.

  Test Case Derivation: Data from the simulation should closely match the experimental data collected from the physical car.

  How test will be performed: The script will run with the provided inputs. After execution is complete, the output of the simulation will be compared to experimental data from the physical car to ensure that the simulation is accurate. The experimental data will be configured to display the acceleration in the direction of travel of the vehicle.

\end{enumerate}

\subsubsection{CVT Dynamics}

This section provides tests to verify the CVT dynamics of the system.
They are based on the following functional requirements from the SRS document.
Listed is each requirement and how the tests verify each one.
\begin{itemize}
  \item [R1:] Verifies the mathematical modeling of the CVT system
  \item [R5:] Verifies the calculations of the primary clamping force
  \item [R6:] Verifies the calculations of the secondary clamping force
  \item [R7:] Verifies the calculations of the sheave acceleration
  \item [R8:] Verifies the calculations of the belt acceleration
  \item [R9:] Verifies the calculations of RPM and engine torque
\end{itemize}

\paragraph{Clamping forces}

\begin{enumerate}
  
  \item{test-1\\}
  
  Control: Manual
            
  Initial State: The state values from Section~\ref{sec:standard_state}.
  
  Input: The input values from Section~\ref{sec:standard_inputs}.
            
  Output: The resulting graphs should show a clamping force against engine angular velocity beginning from $f_\text{clamp\_min}$, which should then rise to a peak before falling during the shifting phase. Suboptimal parameters may lead to a clamping force that is too low or too high, but a similar shape nonetheless. A typical output graph will look similar to the one shown in Figure~\ref{fig:clamping_force_graph}.
  
  Test Case Derivation: This behaviour is resulting from ideal tuning parameters, and may vary depending on inputs. Ramp and helix geometry will both have a major impact on the shape of these graphs, while weight and spring constants will affect the magnitude.
  
  How test will be performed: The script will run with the provided inputs. After execution is complete, the output will be visualized as a graph. Visual inspection will follow against Figure~\ref{fig:clamping_force_graph} for confirmation of expected behaviour.
  
\end{enumerate}

\paragraph{Shift}

\begin{enumerate}
  
  \item{test-1\\}
  
  Control: Manual
            
  Initial State: The state values from Section~\ref{sec:standard_state}.
  
  Input: The input values from Section~\ref{sec:standard_inputs}.
            
  Output: The resulting graphs should show the shift distance against the engine's angular velocity. An ideal graph will look very similar to the leading edge of a square wave. A typical output graph will look similar to the one shown in Figure~\ref{fig:shift_distance_graph}.
  
  Test Case Derivation: This behaviour is resulting from the CVT's clamping forces overcoming the resistive forces just enough such that the CVT begins shifting. Once this has begun, ramp geometry is expected to cause shifting rather than the engine increasing in speed, maintaining a constant power output. With suboptimal tuning parameters passed in, one might observe a slanted shifting portion.
  
  How test will be performed: The script will run with the provided inputs. After execution is complete, the output will be visualized as a graph. Visual inspection will follow against Figure~\ref{fig:shift_distance_graph} for confirmation of expected behaviour.
  
  \item{test-2\\}

  Control: Manual

  Initial State: The state values from Section~\ref{sec:standard_state}.

  Input: Provided experimental data along with the input values from Section~\ref{sec:standard_inputs}, matching those of the vehicle during experimental data collection. Sample experimental data is located here: \href{https://github.com/gr812b/CVT-Simulator/experimental-data/RPM%20PRIM.csv}{Primary RPM.csv}, \href{https://github.com/gr812b/CVT-Simulator/experimental-data/RPM%20SEC.csv}{Secondary RPM.csv} 

  Output: Mean squared error between the simulated shift against engine velocity compared to experimental data. Differences of shape and magnitude will be considered.

  Test Case Derivation: Data from the simulation should closely match the experimental data collected from the physical car. Key points of interest include the shift point and the slope during the shifting phase.

  How test will be performed: The script will run with the provided inputs. After execution is complete, the output of the simulation will be compared to experimental data from the physical car to ensure that the simulation is accurate. The experimental data will be configured to show the primary RPM divided by the secondary RPM on the y-axis, and the primary RPM on the x-axis.

\end{enumerate}

\begin{enumerate}
  
  \item{test-1\\}
  
  Control: Manual
            
  Initial State: The state values from Section~\ref{sec:standard_state}.
  
  Input: The input values from Section~\ref{sec:standard_inputs}.
            
  Output: The resulting graph show show the engine's angular velocity against the vehicles velocity. The output should show a smooth trend upwards starting at $\omega_\text{idle}$ until reaching the shift point, in which it then moves horizontally until finally reaching full shift and bringing the engine's angular velocity to $\omega_\text{max}$. A typical output graph will look similar to the one shown in Figure~\ref{fig:shift_curve}.
  
  Test Case Derivation: This behaviour results from the CVT's shifting behaviour as a whole. An in-depth explanation of the phases can be found beneath Figure~\ref{fig:shift_curve}.
  
  How test will be performed: The script will run with the provided inputs. After execution is complete, the output will be visualized as a graph. Visual inspection will follow against Figure~\ref{fig:shift_curve} for confirmation of expected behaviour.
  
  \item{test-2\\}

  Control: Manual

  Initial State: The state values from Section~\ref{sec:standard_state}.

  Input: Provided experimental data along with the input values from Section~\ref{sec:standard_inputs}, matching those of the vehicle during experimental data collection. Sample experimental data is located here: \href{https://github.com/gr812b/CVT-Simulator/experimental-data/RPM%20PRIM.csv}{Primary RPM.csv}, \href{https://github.com/gr812b/CVT-Simulator/experimental-data/RPM%20SEC.csv}{Secondary RPM.csv} 

  Output: Mean squared error between the simulated shift curve (see Figure~\ref{fig:shift_curve}) and the experimental data. Differences of key points, shape and magnitude will be considered.

  Test Case Derivation: Data from the simulation should closely match the experimental data collected from the physical car. Key points include the shift point, slope during shifting phase, engagement point and the final shift point.

  How test will be performed: The script will run with the provided inputs. After execution is complete, the output of the simulation will be compared to experimental data from the physical car to ensure that the simulation is accurate. The experimental data will be configured to show the primary RPM on the y-axis and the secondary RPM on the x-axis.

\end{enumerate}

\subsubsection{User Interface}

\begin{enumerate}
  \item{test-1\\}
  Initial State: Default input parameters are loaded
            
  Input: Change the input parameters with new values
            
  Output: Input parameters have been updated 
  
  Test Case Derivation: Based on FR10 and FR11, the user should be able to adjust each of the input parameters (within the limitations of the input itself).
  This includes CVT parameters such as weight, ramp geometry, spring rate and spring pretension as well as other parameters such as vehicle weight, driver weight, traction and angle of incline. 
  
  How test will be performed: Upon a fresh launch of the application, manually change each of the input parameters from the default values to a new value. 
             
  \item{test-2\\}
 
  Control: Automatic
            
  Initial State: Application has not been launched
            
  Input: Launch application
            
  Output: User interface for changing the input parameters is displayed
  
  Test Case Derivation: Based on FR12 validate that the system provides the user with a UI to change the input parameters.
  
  How test will be performed: Upon a fresh launch of the software, the system will assert whether the input parameter page is displayed or not. 
  \item {test-3\\}
  
  Control: Automatic
            
  Initial State: Simulation pre loaded with input parameters
            
  Input: Run simulation 
            
  Output: Graphical representation of simulation data is displayed to the user
  
  Test Case Derivation: Based on FR13, validate that the simulation produces output in the form of graphs.
  
  How test will be performed: The test will run the simulation with a set of pre made input parameters, then it will verify that whether or not a graph has been outputted. 
  \item {test-4\\}
  
  Control: 
            
  Initial State:
            
  Input: 
            
  Output: 
  
  Test Case Derivation: 
  
  How test will be performed: 
  \item {test-5\\}
  
  Control: Manual
            
  Initial State: Simulation has completed
            
  Input: Select export data option
            
  Output: simulation data exported successfully
  
  Test Case Derivation: Based on the FR15 validate user can export simulation data after simulation has completed.
  
  How test will be performed: After simulation has completed successfully export the data and verify that it was exported in the correct format to the right location.

  \item {test-6\\}
  
  Control: Manual
            
  Initial State: Application is not installed
            
  Input: Install application files
            
  Output: Applications installs and functions as expected
  
  Test Case Derivation: Based on FR16 and FR17, validate that the application can be installed and ran on different operating systems and types of computers. 
  
  How test will be performed: This test will be performed on two types of computers, a personal laptop and a desktop. On each of those types of computers it will be ran on each of the operating systems, Windows, Linux and MacOS to verify compatability with each OS and type of computer.

  \end{enumerate}
...

\subsection{Tests for Nonfunctional Requirements}

\wss{The nonfunctional requirements for accuracy will likely just reference the
  appropriate functional tests from above.  The test cases should mention
  reporting the relative error for these tests.  Not all projects will
  necessarily have nonfunctional requirements related to accuracy.}

\wss{For some nonfunctional tests, you won't be setting a target threshold for
passing the test, but rather describing the experiment you will do to measure
the quality for different inputs.  For instance, you could measure speed versus
the problem size.  The output of the test isn't pass/fail, but rather a summary
table or graph.}

\wss{Tests related to usability could include conducting a usability test and
  survey.  The survey will be in the Appendix.}

\wss{Static tests, review, inspections, and walkthroughs, will not follow the
format for the tests given below.}

\wss{If you introduce static tests in your plan, you need to provide details.
How will they be done?  In cases like code (or document) walkthroughs, who will
be involved? Be specific.}

\noindent The nonfunctional requirements we will be testing for are accuracy, useability, maintainability, verifiability, understandability and reusability. 
Referencing the SRS document Nonfunctional Requirement 1 (NFR1: Accuracy) and Nonfunctional Requirement 4 (NFR4: Verifiability) will reference (LINK TO TEST ABOVE). 
The Nonfunctional requirements related to useability and understandability (NFR2 and NFR 5 respectively) will reference the useability/understandability survey linked in the appendix.

\subsubsection{Accuracy}

The below test are to verify the accuracy of the system.
They are based on NFR1 from the SRS document.
		
\paragraph{Title for Test}

\begin{enumerate}

\item{test-id1\\}

Type: Functional, Dynamic, Manual, Static etc.
					
Initial State: 
					
Input/Condition: 
					
Output/Result: 
					
How test will be performed: 
					
\item{test-id2\\}

Type: Functional, Dynamic, Manual, Static etc.
					
Initial State: 
					
Input: 
					
Output: 
					
How test will be performed: 

\end{enumerate}

\subsubsection{Useability}

The below test are to verify the useability of the system.
They are based on NFR2 from the SRS document.

\begin{enumerate}

\item{test-id1\\}

Type: Manual
					
Initial State: 
					
Input/Condition: Users within the Primary User role as well as Baja team members are asked to rate how simple the navigation process of the main interface. 
They are asked to rate this on a scale of (1-5) 1 being extremely difficult and 5 being extremely easy with the other options being 4: somewhat easy, 3: neutral and 2: somewhat difficult. 
					
Output/Result: The average output rating from all users is greater than or equal to a 4(somewhat easy or above expectations).
					
How test will be performed: Each user in the test group will be provided with a survey which provides a series of questions and a scale for each option where 1 represents Poor, 2 represents below expectation, 3 represents satisfactory, 4 represents above average and 5 represents excellent.
The average rating will then be calculated and must be above or equal to 4 for each criteria.  

\item{test-id2\\}
  
Type: Manual
            
Initial State: 
            
Input/Condition: Users within the Primary User role as well as Baja team members are asked to rate the features inputting parameters, adjusting parameters, viewing data outputs and saving and exporting data on how easy it was to use each feature.
They are asked to rate this on a scale of (1-5) 1 being extremely difficult and 5 being extremely easy with the other options being 4: somewhat easy, 3: neutral and 2: somewhat difficult. 
            
Output/Result: The average output rating from all users for each listed feature is greater than or equal to a 4(somewhat easy or above expectations).
            
How test will be performed: Each user in the test group will be provided with a survey which provides a series of questions and a scale for each option where 1 represents Poor, 2 represents below expectation, 3 represents satisfactory, 4 represents above average and 5 represents excellent.
The average rating will then be calculated and must be above or equal to 4 for each criteria.  
  
\end{enumerate}

\subsubsection{Maintainability}

The below test are to verify the maintainability of the system.
They are based on NFR3 from the SRS document.

\begin{enumerate}

\item{test-id1\\}

Type: Manual
					
Initial State: Completed system which includes, the finalized mathematical model, GUI and 3D model.  
					
Input/Condition: The 2023 McMaster Baja CVT's configuration. 
					
Output/Result: The amount of time taken and number of lines of code modified using the 2023 CVT.
					
How test will be performed: The test will be manually completed by the teams Quality Assurance Lead, where the amount of time and number of lines of code will be recorded when implementing the changes to correspond to the 2023 CVT.

\end{enumerate}

\subsubsection{Verifiability}

The below test are to verify the verifiability of the system.
They are based on NFR4 from the SRS document.

\begin{enumerate}

\item{test-id1\\}

Type: Functional, Dynamic, Manual, Static etc.
					
Initial State: 
					
Input/Condition: 
					
Output/Result: 
					
How test will be performed: 

\end{enumerate}

\subsubsection{Understandability}

The below test are to verify the understandability of the system.
They are based on NFR5 from the SRS document.

\begin{enumerate}

\item{test-id1\\}

Type: Functional, Dynamic, Manual, Static etc.
					
Initial State: 
					
Input/Condition: Users within the Primary User role as well as Baja team members are asked to rate how clear they found the features and functions within the system. 
They are asked to rate this on a scale of (1-5) 1 being extremely unclear and 5 being extremely clear with the other options being 4: somewhat clear, 3: neutral and 2: somewhat unclear. 
					
Output/Result: The average output rating from all users for each listed feature is greater than or equal to a 4(somewhat clear or above expectations).
					
How test will be performed: Each user in the test group will be provided with a survey which provides a series of questions and a scale for each option where 1 represents Poor, 2 represents below expectation, 3 represents satisfactory, 4 represents above average and 5 represents excellent.
The average rating will then be calculated and must be above or equal to 4 for each criteria.  

\item{test-id2\\}

Type: Manual
					
Initial State: 
					
Input/Condition: Users within the Primary User role as well as Baja team members are asked to rate their understanding of the simulation results and outputs. 
They are asked to rate this on a scale of (1-5) 1 being extremely unclear and 5 being extremely clear with the other options being 4: somewhat clear, 3: neutral and 2: somewhat unclear. 
					
Output/Result: The average output rating from all users for each listed feature is greater than or equal to a 4(somewhat clear or above expectations).
					
How test will be performed: Each user in the test group will be provided with a survey which provides a series of questions and a scale for each option where 1 represents Poor, 2 represents below expectation, 3 represents satisfactory, 4 represents above average and 5 represents excellent.
The average rating will then be calculated and must be above or equal to 4 for each criteria.  

\end{enumerate}

\subsubsection{Reusability}

The below test are to verify the reusability of the system.
They are based on NFR6 from the SRS document.

\begin{enumerate}

\item{test-id1\\}
    
Type: Manual
              
Initial State: Completed system which includes, the finalized mathematical model, GUI and 3D model.  
              
Input/Condition: The 2023 McMaster Baja CVT's configuration. 
              
Output/Result: The amount of time taken and number of lines of code modified using the 2023 CVT.
              
How test will be performed: The test will be manually completed by the teams Quality Assurance Lead, where the amount of time and number of lines of code will be recorded when implementing the changes to correspond to the 2023 CVT.
This will indicate how long it will take for the system to be adapted to future CVT configurations. 
    
\end{enumerate}

\subsection{Traceability Between Test Cases and Requirements}

\wss{Provide a table that shows which test cases are supporting which
  requirements.}

\section{Unit Test Description}

\wss{This section should not be filled in until after the MIS (detailed design
  document) has been completed.}

\wss{Reference your MIS (detailed design document) and explain your overall
philosophy for test case selection.}  

\wss{To save space and time, it may be an option to provide less detail in this section.  
For the unit tests you can potentially layout your testing strategy here.  That is, you 
can explain how tests will be selected for each module.  For instance, your test building 
approach could be test cases for each access program, including one test for normal behaviour 
and as many tests as needed for edge cases.  Rather than create the details of the input 
and output here, you could point to the unit testing code.  For this to work, you code 
needs to be well-documented, with meaningful names for all of the tests.}

\subsection{Unit Testing Scope}

\wss{What modules are outside of the scope.  If there are modules that are
  developed by someone else, then you would say here if you aren't planning on
  verifying them.  There may also be modules that are part of your software, but
  have a lower priority for verification than others.  If this is the case,
  explain your rationale for the ranking of module importance.}

\subsection{Tests for Functional Requirements}

\wss{Most of the verification will be through automated unit testing.  If
  appropriate specific modules can be verified by a non-testing based
  technique.  That can also be documented in this section.}

\subsubsection{Module 1}

\wss{Include a blurb here to explain why the subsections below cover the module.
  References to the MIS would be good.  You will want tests from a black box
  perspective and from a white box perspective.  Explain to the reader how the
  tests were selected.}

\begin{enumerate}

\item{test-id1\\}

Type: \wss{Functional, Dynamic, Manual, Automatic, Static etc. Most will
  be automatic}
					
Initial State: 
					
Input: 
					
Output: \wss{The expected result for the given inputs}

Test Case Derivation: \wss{Justify the expected value given in the Output field}

How test will be performed: 
					
\item{test-id2\\}

Type: \wss{Functional, Dynamic, Manual, Automatic, Static etc. Most will
  be automatic}
					
Initial State: 
					
Input: 
					
Output: \wss{The expected result for the given inputs}

Test Case Derivation: \wss{Justify the expected value given in the Output field}

How test will be performed: 

\item{...\\}
    
\end{enumerate}

\subsubsection{Module 2}

...

\subsection{Tests for Nonfunctional Requirements}

\wss{If there is a module that needs to be independently assessed for
  performance, those test cases can go here.  In some projects, planning for
  nonfunctional tests of units will not be that relevant.}

\wss{These tests may involve collecting performance data from previously
  mentioned functional tests.}

\subsubsection{Module ?}
		
\begin{enumerate}

\item{test-id1\\}

Type: \wss{Functional, Dynamic, Manual, Automatic, Static etc. Most will
  be automatic}
					
Initial State: 
					
Input/Condition: 
					
Output/Result: 
					
How test will be performed: 
					
\item{test-id2\\}

Type: Functional, Dynamic, Manual, Static etc.
					
Initial State: 
					
Input: 
					
Output: 
					
How test will be performed: 

\end{enumerate}

\subsubsection{Module ?}

...

\subsection{Traceability Between Test Cases and Modules}

\wss{Provide evidence that all of the modules have been considered.}
				
\bibliographystyle{plainnat}

\bibliography{../../refs/References}

\newpage

\section{Appendix}

This is where you can place additional information.

\subsection{Symbolic Parameters}

The definition of the test cases will call for SYMBOLIC\_CONSTANTS.
Their values are defined in this section for easy maintenance.

\subsection{Usability Survey Questions?}

\wss{This is a section that would be appropriate for some projects.}

\newpage{}
\section*{Appendix --- Reflection}

\wss{This section is not required for CAS 741}

The information in this section will be used to evaluate the team members on the
graduate attribute of Lifelong Learning.

\input{../Reflection.tex}

\begin{enumerate}
  \item What went well while writing this deliverable? 
  \item What pain points did you experience during this deliverable, and how
    did you resolve them?
  \item What knowledge and skills will the team collectively need to acquire to
  successfully complete the verification and validation of your project?
  Examples of possible knowledge and skills include dynamic testing knowledge,
  static testing knowledge, specific tool usage, Valgrind etc.  You should look to
  identify at least one item for each team member.
  \item For each of the knowledge areas and skills identified in the previous
  question, what are at least two approaches to acquiring the knowledge or
  mastering the skill?  Of the identified approaches, which will each team
  member pursue, and why did they make this choice?
\end{enumerate}

\end{document}