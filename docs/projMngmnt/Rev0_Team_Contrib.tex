\documentclass{article}

\usepackage{float}
\restylefloat{table}

\usepackage{booktabs}

\title{Team Contributions: Rev 0\\\progname}

\author{\authname}

\date{}

\input{../Comments}
%% Common Parts

\newcommand{\progname}{Baja Dynamics} % PUT YOUR PROGRAM NAME HERE
\newcommand{\authname}{Team \#17, Team Name
\\ Grace McKenna
\\ Travis Wing
\\ Cameron Dunn
\\ Kai Arseneau} % AUTHOR NAMES                  

\usepackage{hyperref}
    \hypersetup{colorlinks=true, linkcolor=blue, citecolor=blue, filecolor=blue,
                urlcolor=blue, unicode=false}
    \urlstyle{same}
                                


\begin{document}

\maketitle

This document summarizes the contributions of each team member for the Rev 0
Demo.  The time period of interest is the time between the POC demo and the Rev
0 demo.

\section{Demo Plans}

\wss{What will you be demonstrating}

For the Rev 0 demo we will have a Unity application that can simulate a CVT.

It is comprised of the following components:\\

\textbf{Unity FrontEnd}

\begin{itemize}
  \item An input page that will take in 8 input parameters from the user.
  \item The input page will have default values in each input field based on the cars current settings.
  \item A results page that allows the user to play/pause the simulation.
  \item On the results page there will also be the following features:
  \begin{itemize}
    \item Gauges that will display the current speed and RPM of the car.
    \item A 3D model of the CVT that will animate based on the input parameters.
    \item A mini car to show the position change and incline of the road.
  \end{itemize}
  \item A download button that will allow the user to download the results of the simulation in the form of a CSV.
  \item An input table that displays the user there input parameters.
\end{itemize}

\textbf{Python Backend}

Our demonstration will center around the simulation of the CVT and how the users inputs affect the results.
The user will first use the default values provided, run the simulation and observe the visual results.
Then the user will go back and change the inputs and then see how the results of the simulation change. 

This aims to demonstrate how our simulation models the CVT based on the inputs provided by the user. The backend intends to function as follows:
\begin{itemize}
  \item The python backend will receive the input from the front end and send back a response.
  \item It will then simulate the drivetrain and dynamics of our custom baja vehicle. This includes the engine, transmission, gearbox, wheels, air resistance, gravity and rolling resistance.
  \item Includes an ODE solver to solve a 2nd order ODE given the inputs from the front end and writes all datapoints to a CSV file.
\end{itemize}

\section{Team Meeting Attendance}



\begin{table}[H]
\centering
\begin{tabular}{ll}
\toprule
\textbf{Student} & \textbf{Meetings}\\
\midrule
Total & 1\\
Grace McKenna & 1\\
Travis Wing & 1\\
Cameron Dunn & 1\\
Kai Arseneau & 1\\
\bottomrule
\end{tabular}
\end{table}


\section{Supervisor/Stakeholder Meeting Attendance}

\begin{table}[H]
\centering
\begin{tabular}{ll}
\toprule
\textbf{Student} & \textbf{Meetings}\\
\midrule
Total & 0\\
Grace McKenna & 0\\
Travis Wing & 0\\
Cameron Dunn & 0\\
Kai Arseneau & 0\\
\bottomrule
\end{tabular}
\end{table}

In between the POC and Rev 0 (as of now) there has not been a supervisor meeting. This is mainly due to the fact
that we have been working on the software implemenation of the project and have not had any questions or concerns that we needed to ask our supervisor.
However we do plan on meeting with our supervisor before we present Rev 0.


\section{Lecture Attendance}


\begin{table}[H]
\centering
\begin{tabular}{ll}
\toprule
\textbf{Student} & \textbf{Lectures}\\
\midrule
Total & 1\\
Grace McKenna & 1\\
Travis Wing & 0\\
Cameron Dunn & 0\\
Kai Arseneau & 0\\
\bottomrule
\end{tabular}
\end{table}


\section{TA Document Discussion Attendance}


\begin{table}[H]
\centering
\begin{tabular}{ll}
\toprule
\textbf{Student} & \textbf{Lectures}\\
\midrule
Total & 1\\
Grace McKenna & 1\\
Travis Wing & 1\\
Cameron Dunn & 1\\
Kai Arseneau & 1\\
\bottomrule
\end{tabular}
\end{table}


\section{Commits}

\begin{table}[H]
\centering
\begin{tabular}{lll}
\toprule
\textbf{Student} & \textbf{Commits} & \textbf{Percent}\\
\midrule
Total & 40 & 100\% \\
Grace McKenna & 8 & 20\% \\
Travis Wing & 10 & 25\% \\
Cameron Dunn & 8 & 20\% \\
Kai Arseneau & 14 & 35\% \\
\bottomrule
\end{tabular}
\end{table}


\section{Issue Tracker}

\wss{For each team member how many issues have they authored (including open and
closed issues (O+C)) and how many have they been assigned (only counting closed
issues (C only)) over the time period of interest.}

\begin{table}[H]
\centering
\begin{tabular}{lll}
\toprule
\textbf{Student} & \textbf{Authored (O+C)} & \textbf{Assigned (C only)}\\
\midrule
Grace McKenna & 1 & Num \\
Travis Wing & 5 & Num \\
Cameron Dunn & 0 & Num \\
Kai Arseneau & 17 & Num \\
\bottomrule
\end{tabular}
\end{table}

\wss{If needed, an explanation for the counts can be provided here.}

\section{CICD}

We will utilize CICD to automatically lint, format and test our code. 

For the python backend we will be using the following tools:
\begin{itemize}
  \item [\textbf{flake8}] - A Python linter that checks for PEP8 compliance.
  \item [\textbf{black}] - A Python code formatter that will ensure consistent code style.
  \item [\textbf{unittest}] - Python's built-in testing framework that will be used for unit testing.
  \item [\textbf{coverage}] - A testing framework for Python that will be used for code coverage.
\end{itemize}

\bigskip
\noindent For the Unity C\# frontend we will be using the following tools:
\begin{itemize}
  \item [\textbf{SonarLint}] - C\# linter that checks for code quality and security vulnerabilities.
  \item [\textbf{StyleCop}] - C\# linter that checks for code style and formatting.
  \item [\textbf{UTF}] - A testing framework for C\# that will be used for unit testing.
  \item [\textbf{UTR}] - A testing framework for Unity that will be used for unit testing and code coverage.
\end{itemize}

\end{document}