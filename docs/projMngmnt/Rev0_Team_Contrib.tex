\documentclass{article}

\usepackage{float}
\restylefloat{table}

\usepackage{booktabs}

\title{Team Contributions: Rev 0\\\progname}

\author{\authname}

\date{}

\input{../Comments}
%% Common Parts

\newcommand{\progname}{Baja Dynamics} % PUT YOUR PROGRAM NAME HERE
\newcommand{\authname}{Team \#17, Team Name
\\ Grace McKenna
\\ Travis Wing
\\ Cameron Dunn
\\ Kai Arseneau} % AUTHOR NAMES                  

\usepackage{hyperref}
    \hypersetup{colorlinks=true, linkcolor=blue, citecolor=blue, filecolor=blue,
                urlcolor=blue, unicode=false}
    \urlstyle{same}
                                


\begin{document}

\maketitle

This document summarizes the contributions of each team member for the Rev 0
Demo.  The time period of interest is the time between the POC demo and the Rev
0 demo.

\section{Demo Plans}

\wss{What will you be demonstrating}

\section{Team Meeting Attendance}



\begin{table}[H]
\centering
\begin{tabular}{ll}
\toprule
\textbf{Student} & \textbf{Meetings}\\
\midrule
Total & 1\\
Grace McKenna & 1\\
Travis Wing & 1\\
Cameron Dunn & 1\\
Kai Arseneau & 1\\
\bottomrule
\end{tabular}
\end{table}


\section{Supervisor/Stakeholder Meeting Attendance}

\begin{table}[H]
\centering
\begin{tabular}{ll}
\toprule
\textbf{Student} & \textbf{Meetings}\\
\midrule
Total & 0\\
Grace McKenna & 0\\
Travis Wing & 0\\
Cameron Dunn & 0\\
Kai Arseneau & 0\\
\bottomrule
\end{tabular}
\end{table}

\wss{If needed, an explanation for the counts can be provided here.}

\section{Lecture Attendance}


\begin{table}[H]
\centering
\begin{tabular}{ll}
\toprule
\textbf{Student} & \textbf{Lectures}\\
\midrule
Total & 1\\
Grace McKenna & 1\\
Travis Wing & 0\\
Cameron Dunn & 0\\
Kai Arseneau & 0\\
\bottomrule
\end{tabular}
\end{table}

\wss{If needed, an explanation for the lecture attendance can be provided here.}

\section{TA Document Discussion Attendance}


\begin{table}[H]
\centering
\begin{tabular}{ll}
\toprule
\textbf{Student} & \textbf{Lectures}\\
\midrule
Total & 1\\
Grace McKenna & 1\\
Travis Wing & 1\\
Cameron Dunn & 1\\
Kai Arseneau & 1\\
\bottomrule
\end{tabular}
\end{table}

\wss{If needed, an explanation for the attendance can be provided here.}

\section{Commits}

\begin{table}[H]
\centering
\begin{tabular}{lll}
\toprule
\textbf{Student} & \textbf{Commits} & \textbf{Percent}\\
\midrule
Total & 40 & 100\% \\
Grace McKenna & 8 & 20\% \\
Travis Wing & 10 & 25\% \\
Cameron Dunn & 8 & 20\% \\
Kai Arseneau & 14 & 35\% \\
\bottomrule
\end{tabular}
\end{table}

\wss{If needed, an explanation for the counts can be provided here.  For
instance, if a team member has more commits to unmerged branches, these numbers
can be provided here.  If multiple people contribute to a commit, git allows for
multi-author commits.}

\section{Issue Tracker}

\wss{For each team member how many issues have they authored (including open and
closed issues (O+C)) and how many have they been assigned (only counting closed
issues (C only)) over the time period of interest.}

\begin{table}[H]
\centering
\begin{tabular}{lll}
\toprule
\textbf{Student} & \textbf{Authored (O+C)} & \textbf{Assigned (C only)}\\
\midrule
Grace McKenna & 1 & Num \\
Travis Wing & 5 & Num \\
Cameron Dunn & 0 & Num \\
Kai Arseneau & 17 & Num \\
\bottomrule
\end{tabular}
\end{table}

\wss{If needed, an explanation for the counts can be provided here.}

\section{CICD}

We will utilize CICD to automatically lint, format and test our code. 

For the python backend we will be using the following tools:
\begin{itemize}
  \item [\textbf{flake8}] - A Python linter that checks for PEP8 compliance.
  \item [\textbf{black}] - A Python code formatter that will ensure consistent code style.
  \item [\textbf{unittest}] - Python's built-in testing framework that will be used for unit testing.
  \item [\textbf{coverage}] - A testing framework for Python that will be used for code coverage.
\end{itemize}

\bigskip
\noindent For the Unity C\# frontend we will be using the following tools:
\begin{itemize}
  \item [\textbf{SonarLint}] - C\# linter that checks for code quality and security vulnerabilities.
  \item [\textbf{StyleCop}] - C\# linter that checks for code style and formatting.
  \item [\textbf{UTF}] - A testing framework for C\# that will be used for unit testing.
  \item [\textbf{UTR}] - A testing framework for Unity that will be used for unit testing and code coverage.
\end{itemize}

\end{document}