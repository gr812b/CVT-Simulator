\documentclass{article}

\usepackage{tabularx}
\usepackage{booktabs}
\usepackage{longtable}
\usepackage{graphicx}
\usepackage{pdflscape}
\usepackage{tabularx}

\title{Reflection and Traceability Report on \progname}

\author{\authname}

\date{}

\input{../Comments}
%% Common Parts

\newcommand{\progname}{Baja Dynamics} % PUT YOUR PROGRAM NAME HERE
\newcommand{\authname}{Team \#17, Team Name
\\ Grace McKenna
\\ Travis Wing
\\ Cameron Dunn
\\ Kai Arseneau} % AUTHOR NAMES                   

\usepackage{hyperref}
    \hypersetup{colorlinks=true, linkcolor=blue, citecolor=blue, filecolor=blue,
                urlcolor=blue, unicode=false}
    \urlstyle{same}
                                


\begin{document}

\maketitle

\plt{Reflection is an important component of getting the full benefits from a
learning experience.  Besides the intrinsic benefits of reflection, this
document will be used to help the TAs grade how well your team responded to
feedback.  Therefore, traceability between Revision 0 and Revision 1 is and
important part of the reflection exercise.  In addition, several CEAB (Canadian
Engineering Accreditation Board) Learning Outcomes (LOs) will be assessed based
on your reflections.}

\section{Changes in Response to Feedback}

\plt{Summarize the changes made over the course of the project in response to
feedback from TAs, the instructor, teammates, other teams, the project
supervisor (if present), and from user testers.}

\plt{For those teams with an external supervisor, please highlight how the feedback 
from the supervisor shaped your project.  In particular, you should highlight the 
supervisor's response to your Rev 0 demonstration to them.}

\plt{Version control can make the summary relatively easy, if you used issues
and meaningful commits.  If you feedback is in an issue, and you responded in
the issue tracker, you can point to the issue as part of explaining your
changes.  If addressing the issue required changes to code or documentation, you
can point to the specific commit that made the changes.  Although the links are
helpful for the details, you should include a label for each item of feedback so
that the reader has an idea of what each item is about without the need to click
on everything to find out.}

\plt{If you were not organized with your commits, traceability between feedback
and commits will not be feasible to capture after the fact.  You will instead
need to spend time writing down a summary of the changes made in response to
each item of feedback.}

\plt{You should address EVERY item of feedback.  A table or itemized list is
recommended.  You should record every item of feedback, along with the source of
that feedback and the change you made in response to that feedback.  The
response can be a change to your documentation, code, or development process.
The response can also be the reason why no changes were made in response to the
feedback.  To make this information manageable, you will record the feedback and
response separately for each deliverable in the sections that follow.}

\plt{If the feedback is general or incomplete, the TA (or instructor) will not
be able to grade your response to feedback.  In that case your grade on this
document, and likely the Revision 1 versions of the other documents will be
low.} 

\subsection{SRS and Hazard Analysis}

The following table corresponds to SRS changes in response to feedback.

    \begin{longtable}{|p{4cm}|p{1.5cm}|p{4cm}|p{1.5cm}|}
    \hline
    \textbf{Feedback Item} & \textbf{Source of Feedback} & \textbf{Change Made in Response} & \textbf{Commit Reference} \\
    \hline
    \endfirsthead
    \hline
    \endhead
    \hline
    \endfoot
    \hline
    \endlastfoot
    Functional Requirements do not seem to have measurable/fit criteria associated with them \href{https://github.com/gr812b/CVT-Simulator/issues/38}{Issue \#38} & Peer review  & No change made, Functional requirements were deemed measurable & N/A \\
    \hline 
    Acknowledging possible floating point errors and specifying a tollerance range for error would be beneficial \href{https://github.com/gr812b/CVT-Simulator/issues/39}{Issue \#39} & Peer review  & Added floating point accuracy consideration to \href{https://github.com/gr812b/CVT-Simulator/blob/main/docs/SRS/SRS.pdf}{NFR1}. 
    & \href{https://github.com/gr812b/CVT-Simulator/commit/b737c9dca12893fff6071cc0cf9fcc4c4b4d7a93}{b737c9d} \\
   
    \hline 
    Adding why specifc section were NA \href{https://github.com/gr812b/CVT-Simulator/issues/40}{Issue \#40} & Peer review & No change made, deemed not relevant & N/A\\
    \hline
    Traceability matrix not clear \href{https://github.com/gr812b/CVT-Simulator/issues/41}{Issue \#41} and \href{https://github.com/gr812b/CVT-Simulator/issues/64}{Issue \#64} & Peer review and TA feedback & Traceability Matrices fixed to be more clear and complete 
    & \href{https://github.com/gr812b/CVT-Simulator/commit/f8dc185e533e52c5260595c5b1c8d48bb4db3ff8}{f8dc185}  
    \href{https://github.com/gr812b/CVT-Simulator/commit/d5080cc95e629483f81ea46df03fbcc2f9c454eb}{d5080cc}  
    \href{https://github.com/gr812b/CVT-Simulator/commit/b7db68a977e81676ac26ccea8c662e55feb7d087}{b7db68a} 
    \\
    \hline
    Reusability metric not practical \href{https://github.com/gr812b/CVT-Simulator/issues/42}{Issue \#42} & Peer feedback & No change made as was deemed practical as previous CVT information was available to us. & N/A\\

    \hline 
    Remove unused SRS folders \href{https://github.com/gr812b/CVT-Simulator/issues/55}{Issue \#55}. & Supervisor feedback & Removed used SRS folders from the repository & 
    \href{https://github.com/gr812b/CVT-Simulator/commit/b6e45c22301071eea9360af5a45d437670bcf9a7}{b6e45c2}  \\
    \hline
    Add References \href{https://github.com/gr812b/CVT-Simulator/issues/61}{Issue \#61}. & TA feedback & Added references to SRS where needed. & 
    \href{https://github.com/gr812b/CVT-Simulator/commit/1632c65edd88ce7af866c39e77177310d6423657}{1632c65}  \\
    \hline 
    Discuss standards \href{https://github.com/gr812b/CVT-Simulator/issues/62}{Issue \#62}. & TA feedback & Added standards section. & 
    \href{https://github.com/gr812b/CVT-Simulator/commit/91148c8a8249d9964023fbcab5e4fcf92a940b2c}{91148c8} \\
    \hline
    Missing numerical rational as why certain numbers were chosen in requirements \href{https://github.com/gr812b/CVT-Simulator/issues/65}{Issue \#65}. & TA feedback & Added rationale for the numbers selected. & 
    \href{https://github.com/gr812b/CVT-Simulator/commit/b5b01370c77954ad1f6c25786c1e88eb61ef5a77}{b5b0137} \\
    \hline
    Missing phase in plane of requirements \href{https://github.com/gr812b/CVT-Simulator/issues/66}{Issue \#66}. & TA feedback & Added phase in plan for requirements. & 
    \href{https://github.com/gr812b/CVT-Simulator/commit/43c023e165f82a2f03926e7721ca326effeb52e47}{43c023e} \\
    \hline
    Add description of sheaves \href{https://github.com/gr812b/CVT-Simulator/issues/158}{Issue \#158} & Supervisor feedback & Added sheave description to improve user understanding. & 
    \href{https://github.com/gr812b/CVT-Simulator/commit/e3953ccd60abfd192f6ebd23efa01e5a88108d30}{e3953cc}
    \end{longtable}


    \noindent The following table corresponds to Hazard Analysis changes in response to feedback.

    \begin{longtable}{|p{4cm}|p{1.5cm}|p{4cm}|p{1.5cm}|}
    \hline
    \textbf{Feedback Item} & \textbf{Source of Feedback} & \textbf{Change Made in Response} & \textbf{Commit Reference} \\
    \hline
    \endfirsthead
    \hline
    \endhead
    \hline
    \endfoot
    \hline
    \endlastfoot
    The failure mode of "insufficient frictional forces" has no requirement associated with it according to the FMEA table \href{https://github.com/gr812b/CVT-Simulator/issues/48}{Issue \#48}. & Peer review  & Addressed why one failure mode did not have requirement due to assumptions made. & 
    \href{https://github.com/gr812b/CVT-Simulator/commit/430fb6dd9bbe740c4c952e974f0799712f0055c0}{430fb6d} \\
    \hline
    Assumptions listed outside Assumptions section \href{https://github.com/gr812b/CVT-Simulator/issues/49}{Issue \#49}. & Peer review & No change made as discussing some assumptions in the scope section was relevant. & N/A\\
    \hline 
    Security requirements missing fit criteria \href{https://github.com/gr812b/CVT-Simulator/issues/50}{Issue \#50}. & Peer review & No change made as security requirements were deemed practical and attainable. & N/A\\
    \hline
    Table is missing references for one failure mode \href{https://github.com/gr812b/CVT-Simulator/issues/52}{Issue \#52}. & Peer review & No change made as this was addressed with \href{https://github.com/gr812b/CVT-Simulator/issues/48}{Issue \#48}. & See     \href{https://github.com/gr812b/CVT-Simulator/commit/430fb6dd9bbe740c4c952e974f0799712f0055c0}{430fb6d} \\
    \hline
    Table is missing failure modes and associated safety requirements for User Interface Component \href{https://github.com/gr812b/CVT-Simulator/issues/53}{Issue \#53}. & Peer review & Added a section to address why there were no UI security requirements. & \href{https://github.com/gr812b/CVT-Simulator/commit/bce0b0260aeedc43a905349bb702547b7a445af0}{bce0b02}\\
    \hline
    Should consider cancellation errors under numerical stability \href{https://github.com/gr812b/CVT-Simulator/issues/54}{Issue \#54}. & Peer review & No change as not planned for this revision of our product. & N/A\\
    \hline 
    Add traceability to security requirements in FMEA \href{https://github.com/gr812b/CVT-Simulator/issues/212}{Issue \#212}. & TA feedback & Added traceability to FMEA table for Security Requirements. & 
    \href{https://github.com/gr812b/CVT-Simulator/commit/430fb6dd9bbe740c4c952e974f0799712f0055c0}{430fb6d} \\ 

\end{longtable}


\subsection{Design and Design Documentation}
The following table corresponds to the changes to the Module Guide(MG) and MIS documents. 
\begin{longtable}{|p{4cm}|p{1.5cm}|p{4cm}|p{1.5cm}|}
    \hline
    \textbf{Feedback Item} & \textbf{Source of Feedback} & \textbf{Change Made in Response} & \textbf{Commit Reference} \\
    \hline
    \endfirsthead
    \hline
    \endhead
    \hline
    \endfoot
    \hline
    \endlastfoot
    MG - Section 6 not valueable to the reader \href{https://github.com/gr812b/CVT-Simulator/issues/100}{Issue \#100}. & Peer review  & No change made as section 6 was deemed valueable.& N/A \\
    \hline
    MG - Link back to SRS to reduce duplicate information and add traceability \href{https://github.com/gr812b/CVT-Simulator/issues/101}{Issue \#101}. & Peer review  & Link to SRS was provided.& 
    \href{https://github.com/gr812b/CVT-Simulator/commit/c9472786934ffacbad18ab7b65ed8313451e8ada}{c947278} \\ 
    \hline
    MG - Anticipated changes could be more clear \href{https://github.com/gr812b/CVT-Simulator/issues/103}{Issue \#103}. & Peer review  & No change made as the anticipated change's section was deemed strait forward.& N/A \\
    \hline

    MG - Should the GUI module not make use of initialize module \href{https://github.com/gr812b/CVT-Simulator/issues/104}{Issue \#104}. & Peer review  & No change made as not applicable. & N/A \\
    \hline

    MG - Missing Add timeline and detail about priority of modules being implemented \href{https://github.com/gr812b/CVT-Simulator/issues/222}{Issue \#222}. & TA feedback  & Added details of priority of modules and breakdown of timeline. & 
    \href{https://github.com/gr812b/CVT-Simulator/commit/a3867bbdd214e4d35cc654ba57f8a74de16630c5}{a3867bb} \\
    \hline

    MIS - Some exceptions were left blank. \href{https://github.com/gr812b/CVT-Simulator/issues/102}{Issue \#102}. & Peer review  & Blank exceptions were removed. & 
    \href{https://github.com/gr812b/CVT-Simulator/commit/5c48147b343cc93bd576ef4c185cf6708cb6ad26}{5c48147} \\ 

\end{longtable}

\noindent The following table corresponds to the changes made in the design of the \progname's  design.
\begin{longtable}{|p{4cm}|p{1.5cm}|p{4cm}|p{1.5cm}|}
    \hline
    \textbf{Feedback Item} & \textbf{Source of Feedback} & \textbf{Change Made in Response} & \textbf{Commit Reference} \\
    \hline
    \endfirsthead
    \hline
    \endhead
    \hline
    \endfoot
    \hline
    \endlastfoot
    Add units to the user inputs to improve user understanding. \href{https://github.com/gr812b/CVT-Simulator/issues/153}{Issue \#153}. & Supervisor feedback & Units were added beside inputs. & 
    \href{https://github.com/gr812b/CVT-Simulator/commit/54bafd2bf352567aea47488b067127ec2b9f7a34}{54bafd2} \\ 
    \hline 
    Add ability to upload dxf files for ramp geometry. \href{https://github.com/gr812b/CVT-Simulator/issues/157}{Issue \#153}. & Supervisor feedback & No changes as did not have time to implement. & 
    N/A \\ 
    \hline
    Check for users entering negative inputs  \href{https://github.com/gr812b/CVT-Simulator/issues/154}{Issue \#154}. & Supervisor feedback & Users are not allowed to enter negative numbers. & 
    \href{https://github.com/gr812b/CVT-Simulator/commit/54bafd2bf352567aea47488b067127ec2b9f7a34}{54bafd2} \\ 
    \hline 
    Provide default values in the input fields \href{https://github.com/gr812b/CVT-Simulator/issues/156}{Issue \#156}. & Supervisor feedback & Users are given default values allowing them to quickly tune CVT. & 
    \href{https://github.com/gr812b/CVT-Simulator/commit/54bafd2bf352567aea47488b067127ec2b9f7a34}{54bafd2} \\ 
    \hline 
    Provide user with loading information to improve user feedback \href{https://github.com/gr812b/CVT-Simulator/issues/204}{Issue \#204} & Supervisor feedback & Added a loading bar to provide the user loading progress. & 
    \href{https://github.com/gr812b/CVT-Simulator/commit/f9c304efe46558aaa3f681d34d50404b479b3449}{f9c304} \\ 

\end{longtable}

\subsection{VnV Plan and Report}

The following table corresponds to changes regarding the VnV Plan.
\begin{longtable}{|p{4cm}|p{1.5cm}|p{4cm}|p{1.5cm}|}
    \hline
    \textbf{Feedback Item} & \textbf{Source of Feedback} & \textbf{Change Made in Response} & \textbf{Commit Reference} \\
    \hline
    \endfirsthead
    \hline
    \endhead
    \hline
    \endfoot
    \hline
    \endlastfoot
    Confusion with Validation report extra \href{https://github.com/gr812b/CVT-Simulator/issues/76}{Issue \#76}. & Peer review  & No change made, as one of our extras is a Validation report. & N/A \\
    \hline 
    Repetitive information that can be listed in your development plan \href{https://github.com/gr812b/CVT-Simulator/issues/77}{Issue \#77}. & Peer review & No change as was deemed not repetitive. & N/A \\
    \hline
    Opportunity for more automated tests \href{https://github.com/gr812b/CVT-Simulator/issues/78}{Issue \#78}. & Peer review & No change as not planed for this revision of the product. & N/A \\
    \hline
    More granular items in the SRS verification plan checklist \href{https://github.com/gr812b/CVT-Simulator/issues/85}{Issue \#85}. & Peer review & No change as not planed for this revision of the product. & N/A \\
    \hline
    The checklist format was not consistent. \href{https://github.com/gr812b/CVT-Simulator/issues/86}{Issue \#86}. & Peer review & Checklist format was fixed so that the format of each checklist was consistent.& 
    \href{https://github.com/gr812b/CVT-Simulator/commit/13e3596e7e3ba6fa7feca6426c6248aba93e2b49}{13e3596} \\
    \hline 
    Traceability labels could be shorter \href{https://github.com/gr812b/CVT-Simulator/issues/88}{Issue \#88}. & Peer review  & No change made, as test labels were deemed appropriate.  & N/A \\
    \hline 
    Add traceability to tests in the tables \href{https://github.com/gr812b/CVT-Simulator/issues/214}{Issue \#214}. & TA feedback & Traceability added to the table. & 
    \href{https://github.com/gr812b/CVT-Simulator/commit/c50551feaf252118ed75dcdf9ee6ea1eb685dc0d}{c50551f} \\
    \hline
    Link usability/understandability in the tests and add checklist where applicable \href{https://github.com/gr812b/CVT-Simulator/issues/215}{Issue \#215}. & TA feedback & Added link to usability/understandability survey and added verifiability checklist. & 
    \href{https://github.com/gr812b/CVT-Simulator/commit/ff2bb8c002761baa2ea05bca5b202977b40ac1b8}{ff2bb8c} and \href{https://github.com/gr812b/CVT-Simulator/commit/https://github.com/gr812b/CVT-Simulator/commit/b26ab16316d9372e622200558502ee3e089b03f9}{b26ab16}.\\

\end{longtable}

The following table corresponds to changes regarding the VnV Report.
\begin{longtable}{|p{4cm}|p{1.5cm}|p{4cm}|p{1.5cm}|}
    \hline
    \textbf{Feedback Item} & \textbf{Source of Feedback} & \textbf{Change Made in Response} & \textbf{Commit Reference} \\
    \hline
    \endfirsthead
    \hline
    \endhead
    \hline
    \endfoot
    \hline
    \endlastfoot
    Consolidating all your symbols, abbreviations and acronyms into one document \href{https://github.com/gr812b/CVT-Simulator/issues/201}{Issue \#201}. & Peer review  & No change made, as symbols are referenced in SRS already & N/A \\
    \hline
    Adding expected result clarity explaining the results of your testing, \href{https://github.com/gr812b/CVT-Simulator/issues/202}{Issue \#202}. & Peer review  & No change made, as not enough time to implement. & N/A \\
    
    \hline 
\end{longtable}    
\section{Challenge Level and Extras}

\subsection{Challenge Level}

\plt{State the challenge level (advanced, general, basic) for your project.  Your challenge level should exactly match what is included in your problem statement.  This should be the challenge level agreed on between you and the course instructor.}

\subsection{Extras}

\plt{Summarize the extras (if any) that were tackled by this project.  Extras
can include usability testing, code walkthroughs, user documentation, formal
proof, GenderMag personas, Design Thinking, etc.  Extras should have already
been approved by the course instructor as included in your problem statement.}

\section{Design Iteration (LO11 (PrototypeIterate))}

\plt{Explain how you arrived at your final design and implementation.  How did
the design evolve from the first version to the final version?} 

\plt{Don't just say what you changed, say why you changed it.  The needs of the
client should be part of the explanation.  For example, if you made changes in
response to usability testing, explain what the testing found and what changes
it led to.}

\section{Design Decisions (LO12)}

\plt{Reflect and justify your design decisions.  How did limitations,
 assumptions, and constraints influence your decisions?  Discuss each of these
 separately.}

\section{Economic Considerations (LO23)}

\plt{Is there a market for your product? What would be involved in marketing your 
product? What is your estimate of the cost to produce a version that you could 
sell?  What would you charge for your product?  How many units would you have to 
sell to make money? If your product isn't something that would be sold, like an 
open source project, how would you go about attracting users?  How many potential 
users currently exist?}

\section{Reflection on Project Management (LO24)}

\plt{This question focuses on processes and tools used for project management.}

\subsection{How Does Your Project Management Compare to Your Development Plan}

\plt{Did you follow your Development plan, with respect to the team meeting plan, 
team communication plan, team member roles and workflow plan.  Did you use the 
technology you planned on using?}

\subsection{What Went Well?}

\plt{What went well for your project management in terms of processes and 
technology?}

\subsection{What Went Wrong?}

\plt{What went wrong in terms of processes and technology?}

\subsection{What Would you Do Differently Next Time?}

\plt{What will you do differently for your next project?}

\section{Reflection on Capstone}

\plt{This question focuses on what you learned during the course of the capstone project.}

\subsection{Which Courses Were Relevant}

\plt{Which of the courses you have taken were relevant for the capstone project?}

\subsection{Knowledge/Skills Outside of Courses}

\plt{What skills/knowledge did you need to acquire for your capstone project
that was outside of the courses you took?}

\end{document}