\documentclass{article}

\usepackage{tabularx}
\usepackage{booktabs}

\title{Problem Statement and Goals\\\progname}

\author{\authname}

\date{}

\input{../Comments}
%% Common Parts

\newcommand{\progname}{Baja Dynamics} % PUT YOUR PROGRAM NAME HERE
\newcommand{\authname}{Team \#17, Team Name
\\ Grace McKenna
\\ Travis Wing
\\ Cameron Dunn
\\ Kai Arseneau} % AUTHOR NAMES                  

\usepackage{hyperref}
    \hypersetup{colorlinks=true, linkcolor=blue, citecolor=blue, filecolor=blue,
                urlcolor=blue, unicode=false}
    \urlstyle{same}
                                


\begin{document}

\maketitle

\begin{table}[hp]
\caption{Revision History} \label{TblRevisionHistory}
\begin{tabularx}{\textwidth}{llX}
\toprule
\textbf{Date} & \textbf{Developer(s)} & \textbf{Change}\\
\midrule
Sept 24 & All & Initial draft\\
Apr 3 & Cameron Dunn & Updated to Rev 1\\
\bottomrule
\end{tabularx}
\end{table}

\section{Problem Statement}
The problem statement is a high-level overview of the problem that our project is trying to solve.
It specifies the problem in plain language and outlines the inputs, outputs, stakeholders, and environment.

\subsection{Problem}
The McMaster Baja engineering team is seeking to improve the tuning process for their Continuous Variable Transmission (CVT). The tuning of a CVT typically involves extensive real-world testing of physical components, which is time-consuming and prone to inconsistencies due to factors such as wear and weather. These conditions complicate the ability to fine-tune the CVT’s torque transfer, which directly impacts vehicle acceleration and hill-climbing performance, both which are two of the team's main objectives in competition.


\subsection{Inputs and Outputs}
The inputs and outputs are the parameters and results that define the structure of the problem.
The inputs provide a scope of what factors are being considered for the problem at hand.
The outputs give the desired final form of the inputs after the problem has been solved.

\subsubsection{Inputs}
\begin{itemize}
    \item CVT Parameters
    \begin{itemize}
        \item Primary Flyweight Mass
        \item Primary Spring Pretension
        \item Primary Spring Rate
        \item Primary Ramp Geometry
        \item Secondary Torsional Spring Rate
        \item Secondary Compression Spring Rate
        \item Secondary Spring Rotational Pretension
        \item Secondary Spring Linear Pretension
        \item Secondary Helix Geometry
    \end{itemize}
    \item Vehicle Weight
    \item Driver Weight
    \item Traction
    \item Angle of Incline
    \item Total Distance
\end{itemize}
\subsubsection{Outputs}
\begin{itemize}
    \item Data that will be outputted from the simulation
    \begin{itemize}
        \item Time
        \item Car Velocity
        \item Car Position
        \item Shift Velocity
        \item Shift Distance
        \item Engine Angular Position
        \item Secondary Angular Position
        \item Engine RPM
    \end{itemize}
    \item Graphs of the simulation output
    \item 3D Model of CVT Visualization
    \item Speedometer
    \item Tachometer
    \item Distance travelled visualization
\end{itemize}

\subsection{Stakeholders}
\begin{itemize}
    \item McMaster Baja Racing Team
    \item Dr. Spencer Smith
\end{itemize}

\subsection{Environment}
The environment of the problem encompasses what hardware the application will be able to run on,
as well as the software that the application will support and be developed in.

\subsubsection{Hardware}
\begin{itemize}
    \item Device can be configured through the use of a personal computer or laptop.
\end{itemize}
    \subsubsection{Software}
\begin{itemize}
    \item The application will be supported on Mac, Windows and Linux operating
    systems.
    \item The application will be developed using Python and C\#.
\end{itemize}

\section{Goals}

Develop a 3D model and simulation of the CVT system, which will include the driven pulley, 
driving pulley, belt, and engine, and will be implemented within a graphical user interface.

\subsection{Mathematical Model}
The mathematical model is the main component of the simulation and will be used to 
calculate the output of the CVT system based on the input parameters.
This includes the calculation of the driving pulley's weights, ramps, and spring, 
as well as the driven pulley's helix and spring.
From these, simulating the belt's interaction with both pulleys will allow us to output 
the CVT ratio, RPM, and torque. Finally, combining this with the engine's input
will allow us to calculate the vehicle's acceleration, speed, and distance.

\subsection{Graphical User Interface}
The graphical user interface will allow users to input various parameters such as CVT specifications, 
vehicle and driver weight, traction, and angle of incline. Once complete, it will display a 3D model
of the CVT system and the vehicle, as well as graphs of the simulation output.

\subsection{Data Output and Visualization}
After the simulation is complete, the application will display graphs of the simulation output 
and allow users to export simulation data.

\subsection{Data Validation}
Validation of the simulated data against real-world data is crucial to ensure accuracy. 
We will need to gather data from the McMaster Baja Racing Team to validate against, including 
speed, RPM and basic torque measurements.

\section{Stretch Goals}
The stretch goals of the project are the features that would improve the application,
but they are not required and can be considered out of scope for the current implementation.

\subsection{Simulating Heat}
Heat can play an important role in the behaviour of the CVT system, especially in extreme 
configurations. Including basic heat simulation would allow users to understand which
setups would lead to catastrophic results. Further, minimizing heat can also lead to an optimized
performance.

This goal can be considered complete by including heat readings of various points on the 3D 
model, powered by the simulation of the heat generated by the CVT system.

\subsection{Simulating Inertia of Non Rigid Components}
Include complex simulation of the intertia of non-rigid components such as the belt and 
springs rather than treating them as rigid bodies.

\subsection{Optimization Algorithm}
Implement an optimization algorithm to find the optimal CVT parameters for a given set of 
conditions.

\subsection{Advanced Data Validation}
Depending on the available data, more advanced data validation techniques can be used to 
ensure the accuracy of the simulation. This includes validating advanced torque readings and 
acceleration data.

\section{Challenge Level and Extras}
The challenge level provides a rough estimate of the overall difficulty of the project.
The extras are project specific additional deliverables that separate it from the standard capstone project requirements

\subsection{Challenge Level}

The challenge level for this capstone project is general.

\subsection{Extras}
The extras for this capstone project are:
\begin{itemize}
    \item Usability Testing
    \item Validation Report
\end{itemize}


\newpage{}

\section*{Appendix --- Reflection}

\input{../Reflection.tex}

\begin{enumerate}
    \item What went well while writing this deliverable? 

    While writing this deliverable some of the things that went well included
    defining the problem, inputs, stakeholders, goals and environment. This was clear to us as 
    we've encountered these issues firsthand and have a good understanding of the system.
    
    \item What pain points did you experience during this deliverable, and how
    did you resolve them?

    During the writing of this deliverable, we ran into several pain points. The primary spots 
    ended up being with the outputs, stretch goals and extras. It took more effort than we initially
    thought to determine what the outputs of our application were going to be. This is most likely
    due to the complexity of the system, along with the consideration of what data points would be valuable
    to our stakeholders. Stretch goals became more complex due to the math simulation's complexity. We realized
    certain aspects would be best approximated for our final product rather than risk getting bogged down in 
    the details as we were unsure about the level of accuracy we'd be able to achieve. Finally, we had a hard 
    time deciding on which extras to include in our project. We have went back and forth on which extras 
    to include as part of our project, since we were unfamiliar with the options present.

    As well there were disagreements regarding the scope of the project and the level of detail that should be included in the development plan. 
    Some team members had concerned about the level of difficultly of the project and if we would be able to complete it. On the other hand,
    other team members felt that we were capable of completing the project and needed to add more details for the deliverable.
    We resolved these issues by meeting as a team and doing a walkthrough of the entire project and its components.
    This helped the concerned team members understand the project better and made it feel more attainable. 
    This walkthrough ended up serving as the basis for the development plan.

    There were some disagreements within the team about the project's scope and the level of detail required in the development plan.
    While some members were concerned about the difficulty, others believe we were fully capable and needed to add more detail in our deliverables.
    We resolved these issues by holding a team meeting and conducting a full walkthrough of our system and its components. This helped clarify the project
    for those with concerns, as well as reinforce those who believed in the project. This walkthrough then ended up serving as the foundation for our 
    development plan.
    
    \item How did you and your team adjust the scope of your goals to ensure
    they are suitable for a Capstone project (not overly ambitious but also of
    appropriate complexity for a senior design project)?

    The scope of our project had to be adjusted a bit to ensure it was suitable
    for a capstone project. Originally, it was a bit overambitious as we
    had plans to implement an optimization algorithm to be used to find the
    optimal parameters for the CVT. However, upon flushing out the goals for
    this project, we realized optimization would be better suited as a
    stretch goal rather than one that is required for the completion of the project
    since it could be quite a difficult to achieve.
    \end{enumerate}  

\end{document}