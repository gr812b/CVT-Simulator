\documentclass{article}

\usepackage{tabularx}
\usepackage{booktabs}

\title{Problem Statement and Goals\\\progname}

\author{\authname}

\date{}

\input{../Comments}
%% Common Parts

\newcommand{\progname}{Baja Dynamics} % PUT YOUR PROGRAM NAME HERE
\newcommand{\authname}{Team \#17, Team Name
\\ Grace McKenna
\\ Travis Wing
\\ Cameron Dunn
\\ Kai Arseneau} % AUTHOR NAMES                  

\usepackage{hyperref}
    \hypersetup{colorlinks=true, linkcolor=blue, citecolor=blue, filecolor=blue,
                urlcolor=blue, unicode=false}
    \urlstyle{same}
                                


\begin{document}

\maketitle

\begin{table}[hp]
\caption{Revision History} \label{TblRevisionHistory}
\begin{tabularx}{\textwidth}{llX}
\toprule
\textbf{Date} & \textbf{Developer(s)} & \textbf{Change}\\
\midrule
Sept 19 & Travis & Wrote down Problem, stakeholders and potential extras/challenge level\\
Sept 22 & Travis, Kai & Intial Draft of Problem Statement and Goals\\
... & ... & ...\\
\bottomrule
\end{tabularx}
\end{table}

\section{Problem Statement}

\subsection{Problem}
The McMaster Baja engineering team is seeking to improve the tuning process for their Continuous Variable Transmission (CVT). The tuning of a CVT typically involves extensive real-world testing of physical components, which is time-consuming and prone to inconsistencies due to factors such as wear and weather. These conditions complicate the ability to fine-tune the CVT’s torque transfer, which directly impacts vehicle acceleration and hill-climbing performance, both which are two of the team's main objectives in competition.


\subsection{Inputs and Outputs}

\subsubsection{Inputs}
\begin{itemize}
    \item CVT parameters
    \begin{itemize}
        \item Primary weight
        \item Primary Ramp Geometry
        \item Primary Spring Rate
        \item Primary Spring Pretension
        \item Secondary Helix Geometry
        \item Secondary Spring Rate
        \item Secondary Spring Pretension
    \end{itemize}
    \item Vehicle and Driver Weight
    \item Traction
    \item Angle of Incline
\end{itemize}
\subsubsection{Outputs}
\begin{itemize}
    \item Graphs
    \begin{itemize}
        \item Accleration
        \item Speed
        \item Distance
        \item Clamping Force
        \item CVT Ratio
        \item RPM
        \item Torque
        \item Belt Slippage
    \end{itemize}
\end{itemize}

\subsection{Stakeholders}
\begin{itemize}
    \item McMaster Baja Racing Team
    \item Dr. Spencer Smith
\end{itemize}

\subsection{Environment}
\subsubsection{Hardware}
\begin{itemize}
    \item Device can be configured through the use of a personal computer or laptop.
\end{itemize}
    \subsubsection{Software}
\begin{itemize}
    \item The application will be supported on both Mac, Windows and Linux operating
    systems.
\end{itemize}

\section{Goals}

Develop a 3D simulation of the CVT system, which will include the driven pulley, driving pulley, belt, and engine, and will be implemented within a graphical user interface.
\subsection{Matthematical Model}
    \begin{itemize}
        \item Develop a mathematical model of the CVT system
        \item Implement the mathematical model in the simulation
    \end{itemize}
\subsection{Graphical User Interface}
The graphical user interface will allow users to input various parameters such as CVT specifications, vehicle and driver weight, traction, and angle of incline, while also displaying graphs of the simulation output and enabling the export of simulation data. 
Users will then be able to see the output of there simulation in the form of graphs and export the data for further analysis. 

\subsection{Data Output and Visualization}
After the simulation is complete, the application will display graphs of the simulation output and allow users to export simulation data.
\subsection{Data Validation}
Validation of the simulation data against real-world data is crucial to ensure the accuracy of the simulation.
Will need to gather data from the McMaster Baja Racing Team to validate the simulation data.

\section{Stretch Goals}
\subsection{Simulating Heat}
\subsection{Simulating Inertia of Non Rigid Components}
\subsection{3D model something something}
\subsection{Optimization Algorithm}
Implement an optimization algorithm to find the optimal CVT parameters for a given set of conditions.
\subsection{Advanced Data Validation somethign something}

\section{Challenge Level and Extras}

\subsection{Challenge Level}

The challenge level for this capstone project is General

\subsection{Extras}
The extras for this capstone project are:
\begin{itemize}
    \item Code walkthroughs
    \item Validation Report
\end{itemize}


\newpage{}

\section*{Appendix --- Reflection}

\input{../Reflection.tex}

\begin{enumerate}
    \item What went well while writing this deliverable? 
    \begin{itemize}
        \item Defining problem, inputs, stakeholders, goals and environemnt were very easy. That part was well defined.
    \end{itemize}
    \item What pain points did you experience during this deliverable, and how
    did you resolve them?
    \begin{itemize}
        \item Determing ouputs, stretch goals and extras were hard
    \end{itemize}
        \item How did you and your team adjust the scope of your goals to ensure
    they are suitable for a Capstone project (not overly ambitious but also of
    appropriate complexity for a senior design project)?
    \begin{itemize}
        \item Had to turn some goals into stretch goals to ensure it was suitable for capstone (optimaztion, etc)
    \end{itemize}
\end{enumerate}  

\end{document}